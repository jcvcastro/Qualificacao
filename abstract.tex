% -*- tex-master: "qualificacao.tex" -*-
%!tex root = qualificacao.tex

\chapter*{Abstract}
\addcontentsline{toc}{chapter}{Abstract}

\vspace{-2cm}

In conventional procedures, the design of controllers is based on a mathematical model that represents the process to be controlled. Another approach, which differs in essence from the conventional one, with increasing studies in the last decades, is the design of data-driven controllers (DDC). In the DDC approach, the controller project does not make direct or indirect use of the process model and the entire project is made from data sampled directly from the process. Most of the DDC techniques are iterative methods based mainly on the gradient method to minimize some cost index. However, some techniques, especially the Virtual Reference Feedback Tunning (VRFT), allows, through a batch procedure, to minimize this index using the usual techniques in the scope of systems identification. In general, these procedures are done by setting a structure for the controller, and based on identification methods, the best set of parameters for the controller is identified, aiming to results in a behavior close to that defined by a reference model. In this research, an alternative approach is proposed, in which the objective is to select the best structure of the controller, before the process of tuning or identification of the controller's parameters. In this sense, a method has been developed to assist in the selection of the best controller structure based on a VRFT control strategy for non-linear systems. This selection is made with a modified version of a recent published randomized  model structure selection approach used in model identification. An analise of the structure selection problem is made in sense of controller structure selection and parameter identification. As a qualification text, some preliminary results achieved are studied, and proposals for further work are presented.


proposta

\textbf{Keywords:} Data-Driven Control; Model-Free Control; Reinforcement Learning; Structure Selection; Nonlinear Systems.

% Em procedimentos convencionais o projeto de controladores é feito a partir de um modelo matemático que representa o processo a se controlar.
% Uma outra abordagem, que difere em essência da convencional, com estudos crescentes nas últimas décadas, é a de projeto de controladores baseado em dados (DDC do inglês Data-Driven Control).
% No DDC, o projeto do controlador não faz uso direta ou indiretamente do modelo do processo e todo o projeto é feito a partir de dados amostrados diretamente do processo.
% Grande parte das técnicas DDC são métodos iterativos baseados principalmente no método do gradiente para minimizar algum índice de custo.
% Contudo algumas técnicas, em especial a técnica VRFT (do inglês Virtual Reference Feedback Tuning), permitem, por um procedimento em batelada, realizar a minimização deste índice a partir de técnicas usuais no âmbito da identificação de sistemas.
% No contexto de identificação de sistemas é consenso que a estrutura do modelo – quais regressores o compõem – tem forte influência no desempenho dinâmico.
% Neste sentido esta pesquisa de doutorado tem buscado um método para auxilio na seleção da melhor estrutura do controlador a partir de uma estratégia de controle VRFT para sistemas não-lieares.
% Esta seleção é feita por uma abordagem aleatorizada de seleção de estruturas de modelos já conhecida no âmbito de identificação de sistemas, mas que é, neste trabalho, adaptada para lidar com identificação de controladores.
% Afim de reduzir custo computacional e tornar o procedimento mais viável, o uso de técnicas de aprendizado por reforço é considerado.
% Por fim, o uso de informações auxiliares, que tem mostrado benefícios no contexto de identificação de sistemas, é analisado no contexto da identificação de controladores, por meio de restrições que levam em conta características previamente conhecidas no projeto do controlador.



