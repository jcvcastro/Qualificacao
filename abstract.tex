% -*- tex-master: "qualificacao.tex" -*-
%!tex root = qualificacao.tex

\chapter*{Abstract}
\addcontentsline{toc}{chapter}{Abstract}

\vspace{-2cm}

In conventional procedures, the controller's design is based on a mathematical model that represents the process to be controlled. Another approach, which differs in essence from the conventional one, with growing studies in the last decades, is the design of data-driven controllers, or Data-Driven Control (DDC). In DDC approach, the controller design does not make direct or indirect use of the process model, and the entire project is made from data sampled directly from the process. Most of the DDC techniques are iterative methods based mainly on gradient methods to minimize some cost index. However, some techniques, in particular the Virtual Reference Feedback Tuning (VRFT), allow, through a batch procedure, to minimize this index with usual techniques from systems identification field. In the systems identification context, there is a consensus that the model's structure - which regressors comprise it - has a strong influence on dynamic performance. In this sense, this doctoral research has sought for a method to assist in the selection of the best controller structure tunned by a VRFT control strategy for non-linear systems. The structure selection is made by a randomized model structure selection approach already known in the scope of systems identification, but which, in this work, is adapted to deal with the identification of controllers models. In order to reduce computational cost and make the procedure more viable, the use of reinforcement learning techniques is considered. Finally, the use of auxiliary information, which has shown benefits in the context of system identification, is analyzed in the context of the controllers identification, through restrictions that take into account previously known characteristics in the design of the controller.


\textbf{Keywords:} Data-Driven Control; Model-Free Control; Reinforcement Learning; Structure Selection; Nonlinear Systems.

% Em procedimentos convencionais o projeto de controladores é feito a partir de um modelo matemático que representa o processo a se controlar.
% Uma outra abordagem, que difere em essência da convencional, com estudos crescentes nas últimas décadas, é a de projeto de controladores baseado em dados (DDC do inglês Data-Driven Control).
% No DDC, o projeto do controlador não faz uso direta ou indiretamente do modelo do processo e todo o projeto é feito a partir de dados amostrados diretamente do processo.
% Grande parte das técnicas DDC são métodos iterativos baseados principalmente no método do gradiente para minimizar algum índice de custo.
% Contudo algumas técnicas, em especial a técnica VRFT (do inglês Virtual Reference Feedback Tuning), permitem, por um procedimento em batelada, realizar a minimização deste índice a partir de técnicas usuais no âmbito da identificação de sistemas.
% No contexto de identificação de sistemas é consenso que a estrutura do modelo – quais regressores o compõem – tem forte influência no desempenho dinâmico.
% Neste sentido esta pesquisa de doutorado tem buscado um método para auxilio na seleção da melhor estrutura do controlador a partir de uma estratégia de controle VRFT para sistemas não-lieares.
% Esta seleção é feita por uma abordagem aleatorizada de seleção de estruturas de modelos já conhecida no âmbito de identificação de sistemas, mas que é, neste trabalho, adaptada para lidar com identificação de controladores.
% Afim de reduzir custo computacional e tornar o procedimento mais viável, o uso de técnicas de aprendizado por reforço é considerado.
% Por fim, o uso de informações auxiliares, que tem mostrado benefícios no contexto de identificação de sistemas, é analisado no contexto da identificação de controladores, por meio de restrições que levam em conta características previamente conhecidas no projeto do controlador.



