%!TEX root = ../Qualificacao.tex

%%%%%%%%%%%%%%%%%%%%%%%%%%%%%%%%%%%%%%%%%%%%%%%%%%%%%%%%%%%%%%%%%%%%%%%
%                 EXAMPLE 4.1 (51) - 2nd Order System                 %
%%%%%%%%%%%%%%%%%%%%%%%%%%%%%%%%%%%%%%%%%%%%%%%%%%%%%%%%%%%%%%%%%%%%%%%


% exemplo 51
\begin{exmp}[Second order Linear System -- continued.]\label{ex:sis2aord}

  Consider the 2nd-order linear system describe in Example \ref{exm:31}: 
  \begin{equation}
    \label{eq:sis2aord}
    y_k = a_1y_{k-1} + a_2y_{k-2} + b_1u_{k-1} + b_2u_{k-2}.
  \end{equation}
  % onde, para $i = 1,\ 2$, os termos $a_i$, $b_i$ $\in \R$, representam parâmetros constantes; $u_{k-i}$ e $y_{k-i}$ $\in \R$, os sinais de entrada e saída, respectivamente; e $k$ o índice temporal.
  According to the VRFT strategy (Chapter ~\ref{cap:VRFT}), one must first define a class of allowable controllers $\mathscr{C}$, containing the desired  structure, and define a reference model that expresses the desired close loop system's behavior.

  In this example, we want to analyze the use of the RAmSS algorithm directly, without modifications, in order to find the best structure for a controller designed from the VRFT strategy.
  In order to examine the behavior of the RaMSS Algorithm in finding the best set of regressors, we proceed as follows.

  A feasible reference model is defined, that is, with a relative degree equal to or greater than that of the process. For simplicity, a 1st order model with the same relative degree of the process is chosen, as Example \ref{exm:31}, given by the transfer function
  \begin{equation}
    T_d(z) = \frac{1-A}{z-A},
    \label{eq:mr_sis2aord}
  \end{equation}
  The parameter adopted here is the same as Exemple \ref{exm:31}, i.e. $A = -T_s/\tau_d = 0.8187$. 
  The ideal controller, represented by an ARX model, will be given by the structure composed of the regressors presented in \eqref{eq:exp31_uk}, represented here by
  \begin{equation}
    \label{eq:exp51_contIdeal}
    u_k = \theta_0e_{k} + \theta_1e_{k-1} + \theta_2e_{k-2} + \theta_3u_{k-1} + \theta_4u_{k-2},
  \end{equation}
  and will have the parameters shown in \eqref{eq:exp31_ideal_parameters}, described here as
  \begin{equation}
    \vtheta^\star= \begin{bmatrix} \theta^\star_0 & \theta^\star_1 & \theta^\star_2 & \theta^\star_3 & \theta^\star_4 \end{bmatrix}^T =  \begin{bmatrix} 0.181 & 0.308 &  0.145 &  0.19 & 0.81 \end{bmatrix}^T.
  \label{eq:ex51_ideal_parameters}
\end{equation}
Using the RaMSS algorithm, as discussed in the section ~\ref{sec:ramss}, the procedure is performed for 2 cases:
\begin{description}
  \item[case A] the universe set $\mathscr{M}$ is taken as all possible 3rd degree non linear models formed by the monomials up to 4 delays for input $\tilde{e}_k$ (virtual error) and output $\tilde{u}_k$ (virtual process input) collected data. No noise is considered in this case.
  \item[case B] the same case as case A, but now with a noise in the output, given by a gaussian distribution, i.e. $\nu \sim \mathcal{N}(\mu,\sigma)$, where $\mu=0$ is the mean, and $\sigma= 0.1$ is the standard deviation adopted. Note that the noise is added in the output, in a way that the model for the process is represented by an output error model (OEM).
\end{description}
The RaMSS procedure is applied to a data set of 700 samples, obtained via the VRFT procedure, in which the VRFT filter is considered to be unitary, i.e. the data is not filtered. The same data set is applied for the 2 cases. At each iteration, 100 candidate models are randomly chosen to be analyzed, using a Bernoulli process. The RIPs are updated in a maximum of 100 iterations, or until they all converge to a margin above 0.9 or below 0.1, i.e. the following stopping criterion is adopted:
\begin{align}
  \textbf{if } \left[iter<iter_{\max}\right] \textbf{ or } \left[\Delta_S < \frac{1}{2} \left(1-\min_{\forall \mu \in \bm{\mu}_k}{ \left| 2\mu -1 \right| }\right)\right] \textbf{ then} \text{, STOP the procedure.}
\label{eq:crit.par}
\end{align}
where $\Delta_S = 0.1$ is the convergence margin, $iter$ is the iteration number and $iter_{\max} = 100$ is the maximum allowed iterations.


Applying the procedure in both cases, the following controllers are obtained, for a specific realization:
\begin{align}
  % Para caso 24:
  \label{eq:ex51CasesAB}
  u_1(k) &= 0.636{u}_1(k-1) + 0.513{u}_1(k-2) + -0.321{u}_1(k-3) + 0.17{u}_1(k-4) \nonumber\\
      &\quad+ 0.181{e}_1(k) + 0.227{e}_1(k-1) + 0.045{e}_1(k-2) + 0.03{e}_1(k-4) \\
  u_2(k) &= 0.737{u}_2(k-1) + 0.242{u}_2(k-2) + -0.329{u}_2(k-3) + 0.32{u}_2(k-4) \nonumber\\
         &+ 0.175{e}_2(k) + 0.197{e}_2(k-1) + 0.049{e}_2(k-2) + 0.049{e}_2(k-3) + 0.059{e}_2(k-4)
  % u_1(k) &= 0.193{u}_1(k-1) + 0.805{u}_1(k-2)  \nonumber \\
         % &\quad + 0.001{u}_1(k-3) + 0.181{e}_1(k) + 0.307{e}_1(k-1) + 0.144{e}_1(k-2) \nonumber \\
  % u_2(k) &= 0.737{u}_2(k-1) + 0.242{u}_2(k-2) + -0.329{u}_2(k-3) + 0.32{u}_2(k-4)  \\
         % &\quad + 0.175{e}_2(k) + 0.  97{e}_2(k-1) + 0.049{e}_2(k-2) + 0.049{e}_2(k-3) + 0.059{e}_2(k-4)  \nonumber
  % u_1(k) &= 0.193{u}_1(k-1) + 0.805{u}_1(k-2) + 0.001{u}_1(k-3) \\
         % &+ 0.181{e}_1(k) + 0.307{e}_1(k-1) + 0.1441{e}(k-2), \\
  % u_2(k) &= 0.751{u}_2(k-1) + 0.241{u}_2(k-2) -0.33{u}_2(k-3) + 0.322{u}_2(k-4) \\
         % &+ 0.174{e}_2(k) + 0.197{e}_2(k-1) + 0.048{e}_2(k-2) + 0.049{e}_2(k-3) + 0.0591,
\end{align}
where the index subscribed to the variables represent the respective cases.

The table ~\ref{tab:exp51_param} sumarizes the simulation parameters used in \ref{alg:RaMSS}, for the 2 cases,
where $o$ is the maximum allowed degree for the regressors, $n_{\tilde{e}}$ is the maximum delay for the virtual error signal $\tilde{e}(k)$, $n_{\tilde{u}}$ is the maximum delay for the input signal of the sampled plant, $ N_p$ is the number of models chosen at each update of the RIPs, $ iter_{\max} $ is the maximum number of allowed iterations, $\Delta_S$ is the trashold for convergence of RIPs, $K$ is the gain for the performance indexes presented in \ref{eq:Js}, $\gamma_0$ is the initial gain of \ref{eq:gamma}, $\mu_{\min}$ and $\mu_{\max}$ are the minumum and maximum values allowed for the RIPs and $\nu$ is the noise added to the output.
\begin{table}[htpb]
  \centering
  \caption{Parameters for simulating the RaMSS algorithm of the example ~\ref{ex:sis2aord}}\label{tab:exp51_param}
  \begin{tabular}{c|c|c|c|c|c|c|c|c|c|c}
    Case & $o$ & $n_{\tilde{e}}$ & $n_{\tilde{u}}$ & $ N_p$ & $ iter_{\max} $ & $K$ & $\gamma_0$ &  $\mu_{\min}$ & $\mu_{\max}$ & $\nu$\\
    \hline
    A & $ 3 $ & $4$ & $4$ & $100$ & $100$ & $1$ & $2$ & $0.05$ & $1$ & $0$ \\
    B & $ 3 $ & $4$ & $4$ & $100$ & $100$ & $1$ & $2$ & $0.05$ & $1$ & $\mathcal{N}(\mu,\sigma)$
  \end{tabular}
\end{table}\\
\todo[inline]{MEXER NESTA TABELA AINDA! Valores estão errados. Já os modifiquei. Definir parâmetros, etc. Colocar os mais essenciais por aqui e os menos, no appendice. }

Note que os controladores apresentados em \eqref{eq:ex51CasesAB} são obtidos para uma realização específica do procedimento. São considerados todos os termos que terminam com o RIP acima de 0,95. 
A Figura \ref{fig:ex51_RIPevol_2cases} mostra a evolução dos RIPs para esta realização. 

\begin{figure}[H]
  \centering
  \includegraphics[width=\textwidth]{Figs/Cap5/ex51_rips_evol_2cases.tex.pdf}
  \caption{Typical evolution of RIPs for choosing regressors for case 1 and 2.}
  \label{fig:ex51_RIPevol_2cases}
\end{figure}
\todo[inline]{Trocar esta figura} 

Nota-se que os regresores ideias são selecionados antes dos primeiros 40 passos.
Mas ao continuar com as iterações, os regressores $\tilde{u}(k-3)$, $\tilde{u}(k-4)$ e $\tilde{e}(k-4)$, continuam aumentando monotonicamente seu valor e eventualmente acabam sendo selecionados.
O efeito da inclusão destes termos apesar de, em geral serem pequenos, deterioram o comportamento desejado do controlador.
Quando considerado ruído na medição (caso 2), a seleção de regressores não ideais ocorrem ainda mais cedo, como pode ser observado no gráfico inferior da \ref{fig:ex51_RIPevol_2cases}.
Parte deste resultado se deve principalmente ao piora na identificação paraétrica durante o procedimento VRFT.
Uma melhora pode ser obtida aumentado-se o nũmero de amostras do processo.

A Figura~\ref{fig:Figs-ResponderSist2aordNARX-png} mostra a resposta temporal para os controladores identificados em \eqref{eq:ex51CasesAB}, quando o sinal de refeência é tomado como uma onda quadrada.

% The figure  shows the temporal response for a square wave in reference signal, for the 2 cases.
% Note que no Caso 1,

\begin{figure}[H]
  \sbox0{\blacksolidlinethin} \sbox1{\bluedashedline} \sbox2{\reddottedline} \sbox3{\blackdottedline} \sbox4{\bluedashdotedline} 
  \centering
  % \includegraphics[width=1\textwidth]{./Figs/Cap5/ex51_resp_temporal_mf_editado.tex.pdf}
  \includegraphics[width=1\textwidth]{./Figs/Cap5/ex51_resp_temporal_mf2.tex.pdf}
  % \include{./Figs/Cap5/ex51_resp_temporal_mf.tex}
  % \caption{Resposta do sistema em malha fechada (gráfico superior) e respectivos erros absolutos (gráfico inferior) utilizando os controladores identificados no caso A (\usebox1) e no caso B (\usebox2). Os sinais de referência (\usebox3) e de reposta do modelo de de referência (\usebox0) são mostrados no gráfico superior. O erro para caso considerando somente a estrutura ideal é representado por (\usebox4), no gráfico inferior.}
  \caption{Closed-loop system response (upper graph) and respective absolute errors (lower graph) using the controllers identified in case A (\usebox1) and case B (\usebox2). The reference (\usebox3) and the reference model response (\usebox0)  signals are shown in the upper graph. The error for case considering only the ideal structure is represented by (\usebox4), in the lower graph.}
  \label{fig:Figs-RespostaSist2aordNARX-png}
\end{figure}

Nota-se que o comportamento do caso com ruído é deteriorado em relação ao caso 1 (sem ruído). O erro em regime permanente para o caso 2 é da ordem de 10 vezes maior que o caso 1, conforme o gráfico inferior da Figura \ref{fig:Figs-RespostaSist2aordNARX-png} (atenção às escalas diferentes no gráfico de erros da figura).
Esse erro maior se deve a dois fatores: seleção de uma estrutura sobreparametrizada do modelo pelo procedimento RaMSS, e pior identificação paramétrica devido a presença do ruído na saída, que pode resultar em polarização dos parâmetros.
Note que, como o ruído é acrescentado a saída do processo (OEM), e pelo procedimento de filtragem pela inversa da planta do ao se aplicar o VRFT, o ruído, apesar de branco, pode acarretar em polarização nos parâmetros identificados.
O gráfico de erros da Figura \ref{fig:Figs-RespostaSist2aordNARX-png} mostra também a resposta para o caso sem ruídos, mas considerando que apenas os regressores ideais são tomados no processo de identificação.
Neste caso o erro diminui, e diferença entre os erros é atribuída à sobreparametrização provocada pelos regressores extras selecionados no caso 1. 
Apesar disso, um pequeno erro em regime permanente permanece, mesmo sem ruídos. Este fato é atribuído ao efeito discutido no Capítulo \ref{cap:VRFT}, em que pequenos erros na identificação de parâmetros fazem com que a soma dos coeficientes em $y(k)$ não seja exatamente nula resultando em um sistema com alto ganho em baixas frequências, mas não com ganho infinito. Como discutido, no capítulo \ref{cap:VRFT} e a ser proposto no capítulo \ref{cap:Concl}, espera-se que este problema possa ser resilvido com a imposição de restrições no processo de identificação (uso de informação auxiliar).





\todo[inline]{Ainda em construção, por enquanto só coloquei alguns dos gráficos que irei utilizar. Comparações, com tabelas comparando resultados do RaCSS com o ERR também serão ainda colocados. Logicamente, com devidas análises.}

\end{exmp}




