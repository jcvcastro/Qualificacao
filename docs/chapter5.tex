%!TEX root = Qualificacao.tex

%%%%%%%%%%%%%%%%%%%%%%%%%%%%%%%%%%%%%%%%%%%%%%%%%%%%%%%%%%%%%%%%%%%%%%%
%                              R a C S S                              %
%%%%%%%%%%%%%%%%%%%%%%%%%%%%%%%%%%%%%%%%%%%%%%%%%%%%%%%%%%%%%%%%%%%%%%%
% 2021-05-08

\chapter{Randomized Controller Structure Selection}\label{cap:CCS}
\vspace{-1cm}

% Frase citação inicial {{{1
% \begin{flushright}
% \begin{minipage}{0.7\linewidth}
%     \emph{``\dots''}
% \end{minipage}
% \end{flushright}
%
% \begin{flushright}
% Cicrano
% \end{flushright}
%
%\vspace{1cm}

% Intro CAP {{{2

In this chapter, an approach for the design of DDC controllers is presented, based on the RamSS methodology for the choice of structure and on the VRFT method for tuning the controller. This new approach is here called Randomized Controller Structure Selection, or RaCSS, for short.
% In this chapter, proposals for the design of DDC controllers, with regard to the choice of controller structure are presented. The methodology used, as well as the preliminary results obtained in a first stages of this research are presented.

In this research, focus has been given to the study and development of structure selection techniques for the model of feedback controllers in a DDC\@ design fashion.
The intention is, at first, in the scope of non-linear control systems, to adapt known techniques of structure selection in the area of systems identification to deal with the control case, where the system to be identified becomes the controller. To achieve these objectives, it was decided, to use polynomial models to represent the controllers, initially with NARX structures. This choice is convenient due to the structural flexibility characteristic and the ability of these models to describe non-linear systems~\citep{pearson1999, martins2013}.
% \begin{figure}[htpb]
% \centering
% \missingfigure{Pretendo colocar por aqui um diagramas (deve ser blocos) para ajudar.}
% % \includegraphics[width=0.8\textwidth]{}
% \caption{}
% \label{fig:Diagrama representativo}
% \end{figure}
%

As mentioned in Chapter~\ref{cap:cap2}, one of the main steps in the system identification procedure from data is the step of choosing the model structure. Works that aim to help on this task have been proposed, as is the case of the papers published  by~\cite{baldacchino2012,baldacchino2013} and \cite{falsone2014,falsone2015},
\todo{Vou colocar aqui mais referências neste sentido, de seleção de estruturas} 
in which a random approach is used in order to select the best candidates among the possible regressors in the formation of the model to be identified.

% As mentioned in Chapter~\ref{cap:cap2}, one of the main steps in the procedure of model system identification from data is the model selection step.  Studies that aim to deal with this task have been proposed, as is the case of studies developed by~\cite{falsone2014, falsone2015} in which a random approach, named RaMSS, is used in order to select the best candidates to compose a final model structure, among a class of possible regressors.
%
% \todo{Confirmar se o RaMSS original só é aplicável a NARX mesmo na sua forma original. Se for, falar aqui.}
More recently, ~\cite{retesNARMAXModelIdentification2019} extended the RaMSS strategy for choosing NARMAX model structures, which take into account the effect of noise on signals in order to reduce the polarization in parametric estimates.

In the scope of controller design based on data, more specifically by the VRFT method, in which the controller parameters are adjusted by conventional identification techniques such as OLS, ILS, IV, among others, the same problem of choosing the model structure occurs, with the difference that the model now represents the controller, and no longer the process. Therefore, analogies must be able to be made intended to use the RaMSS method in order to extend it for the purpose of identifying the most suitable structure for the controller.

In this sense, the present work has been directed to the study of the possibilities and consequences of using these technologies for control purposes, aiming to choose the most suitable controller structure.

The next section presents the basics of the methodology adopted for the RaCSS procedure, and then, some preliminary results obtained in the first stages of this research are studied.

\todo[inline]{Pretendo colocar um paragrafo aqui comentando o que será apresentado no restante do capítulo.} 


\section{Methodology}\label{sec:CSS_metod} \todo{Talvez colocar o Título como: ``The RaCSS Methodology''"?}
\todo[inline]{A terminar... Vou mexer no início e fim desta seção ainda, enquanto estiver traduzindo.} 

% Visando atingir o objetivo de se utilizar uma estratégia para estrutura e identificação paramétrica de controladores a partir de uma abordagem DDC, pretende-se neste trabalho, fazer uso das duas metodologias abordadas em capítulos anteriores: o VRFT, para sintonia de parâmetros do controlador e o RaMSS, para seleção da melhor estrutura do controlador.
In order to achieve the objective of using a strategy for structural and parametric identification of controllers from a DDC approach, this work intends to make use of the two methodologies covered in previous chapters: VRFT, for tuning controller parameters and RaMSS, to select the best controller structure.
% A partir de do estudo e adaptações de trabalhos anteriores \citep{retesNARMAXModelIdentification2019}, tem-se desenvolvido um algoritmo capaz de unir as duas estratégias com o objetivo de auxiliar tanto na seleção de estruturas, quando na sintodia de controladores não-lieares (ou lineares).
From the study and adaptations of previous works \citep{retesNARMAXModelIdentification2019}, an algorithm has been developed, capable of uniting the two strategies in order to assist in the selection of structures, as in the tuning of non-linear (or linear) controllers.
% O algoritmo tem servido de  framework para estudo e validação da abordagem proposta, a partir de simuações numéricas.
The algorithm has served as a framework for the study and validation of the proposed approach, based on numerical simulations.
% Neste sentido, parte dos esforços da pesquisa tem se voltado para a implemetação numérica deste framework.
In this sense, part of the research efforts has turned to the numerical implementation of this framework.
% Em uma primeira etapa, foi analisado o comportamento do algoritmo na identificação de modelos de processos, a partir do algoritmo RaMSS, como proposto originalmente.
In a first step, the behavior of the algorithm in the identification of process models was analyzed, using the RaMSS algorithm, as originally proposed.
% Alguns dos resultados de validação desta etapa são apresentados no Capítulo \ref{cap:cap2}, sob a forma do Exemplo \ref{ex:varHeater}.
Some of the validation results for this step was presented in Chapter \ref{cap:cap2}, in the form of Example \ref{ex:varHeater}.


% Uma vez validado o funcionamento do algorítmo, o método VRFT, discutido no Capítulo \ref{cap:VRFT} é acrescentado ao processo, de forma que o desempenho do RaMSS em selecionar estruturas para um controlador possa ser avaliado na prática.
Once the behavior of the algorithm has been validated, the VRFT method, discussed in Chapter \ref{cap:VRFT} is added to the process, so that the performance of RaMSS in selecting structures for a controller can be evaluated in practice.
% Nesta etapa, o algoritmo RamSS, utilizado no Capítulo \ref{cap:cap2} é ligeiramente modificado para lidar com a esimação paramétrica via VRFT, para primeiros testes de projeto de controladores.
In this step, the RaMSS algorithm, presented in Chapter \ref{cap:cap2} is slightly modified to deal with parametric estimation via VRFT, for the first tests of controller design.
% Neste cenário se espera alguns primeiros resultados, sem maiores compromissos com bons desempenhos, mas de são de grande valia para análise e comparação das modificações propostas. Alguns destes resuldados são apresentados pelos exemplos da seção \ref{sec:prel_results} a seguir, mais especificamente no exemplo \ref{ex:51}.
In this scenario, some first results are expected, without major commitments with good performances, but they are of great value for analysis and comparison of the proposed modifications. Some of these results are presented by the examples in the Section~\ref{sec:prel_results} below, more specifically in the Example~\ref{ex:sis2aord}.

% Como abordado no Capítulo \ref{cap:VRFT}, sob certas condições, o controlador pode ser identificado por procedimentos tradicionais de identificação de sistemas pelo método VRFT.
As discussed in Chapter \ref{cap:VRFT}, under certain conditions, the controller can be identified by traditional systems identification procedures using the VRFT method.
% Caso as condições necessárias não sejam satisfeitas (vide a Assumption \ref{ass:machedControl} -- matched control) um filtro pode ser projetado de forma que a minimização do índice de custo relativo aos sinais de entrada e saída do controlador, dado por $J_{VR}(\bm{\theta})$ resulte também na minimização do índice de custo $J_{MR}(\bm{\theta})$ relativo ao erro de rastreamento de um modelo de referência.
If the necessary conditions are not met (see Assumption \ref{ass:machedControl} - matched control) a filter can be designed so that the minimization of the cost index related to the input and output signals of the controller, given by $J_{VR}(\bm{\theta})$, also results in the minimization of the $J_{MR}(\bm{\theta})$ cost index related to the tracking error of a reference model.
% Esta relação é garantida pelo filtro desde que a estrutura do modelo não seja muito ``distante'' da estrutura ideal, i.e. desde que não seja muito subparametrizada.
This relationship is guaranteed by the filter as long as the model structure is not too ``distant'' from the ideal structure, i.e. as long as it is not too under-parameterized.

% Uma proposta da presente pesquisa é utilizar o procedimento RaMSS para tentar identificar a estrutura ideal para o controlador, ou pelo menos uma estrutura próxima que gere bons resultados ao se utilizar o procedimento VRFT. A identificação desta estrutura é feita a partir de dados colhidos sob a abordagem VRFT. Na tentativa de validar a proposta simulações são feitas em diferentes cenários.
A proposal of the present research is to use the RaMSS procedure to try to identify the ideal structure for the controller, or at least a close structure that generates good results when using the VRFT procedure. The identification of this structure is made from data collected under the VRFT approach. In an attempt to validate the proposal, simulations are carried out in different scenarios.
% Em um primeiro cenário o procedimento RaMSS é aplicado a um processo linear onde seu modelo e o modelo do controlador ideal são previamente conhecidos. O intuito aqui é analisar o comportamento do procedimento na seleção de regressores candidatos a um eventual controlador. Uma análise ao estilo Monte Carlo é aplicada ao processo em 2 casos: no primeiro, o procedimento é realizado sem influência de ruídos nos dados utilizados; no segundo, ruídos são acrescentados aos dados no intuito de simulação de um ambiente mais realista. O Exemplo \ref{ex:sis2aord} analisa estes casos.
In a first scenario, the RaMSS procedure is applied to a linear process where their model and the model of the ideal controller (matched control) are previously known.
The aim here is to analyze the behavior of the procedure in the selection of candidate regressors for a possible controller.
A randomized analysis is applied to the process in 2 cases: in the first, the procedure is performed without the influence of noise on the data used; in the second, noise is added to the data in order to simulate a more realistic environment.
Example \ ref {ex: sis2aord} analyzes these cases.

% O procedimento RaMSS foi originalmente proposto para lidar com identificação de processos, e não controladores.
The RaMSS procedure was originally proposed to deal with process identification, not controllers.
% A atualização do vetor de probabilidades para inclusão dos regressores (RIPs) é feita baseando-se em índices de desempenho visando avaliar a qualidade de predição do modelo.
The update of the vector of the Regressor Inclusion Probabilities (RIPs) is done based on performance indexes that evaluate the model's prediction quality.
% Esta predição é feita sobre dados de treinamento, utilizando uma versão exponencial do MSPE, e possivelmente, no intuito de aumentar a robustez, sobre dados de validação (free run simulation) por meio de uma versão exponencial do MSSE.
This prediction is made on training data, using an exponential version of MSPE, and possibly, in order to increase robustness on validation data (free run simulation), through an exponential version of MSSE.
% No entanto, para fins de controle, tais índices podem não ser os melhores.
However, for control purposes, these indexes may not be the best.
% Principalmente o índice baseado no MSSE, que exige maior custo computacional sob a proposta de promover melhoras na robustez do modelo identificado. O efeito é uma melhor predição do sinal de saída do do modelo identificado, mesmo quando excitado com sinais diferentes daqueles usados na identificação.
Mainly the index based on MSSE, which requires higher computational cost under the proposal to promote improvements in the robustness of the identified model. The effect is a better prediction of the output signal of the identified model, even when excited with signals different from those used in the identification.
% Apesar de carecer de mais estudos, a princípio este índice não parece trazer muitos benefícios quando utilizado para fins MSS do controlador, uma vez que o intuito final, no projeto de um controlador, é fazer com que o sinal de saída do processo (e não do controlador), tenha um comportamento desejado.
Despite the lack of further studies, at first, this index does not seems to bring many benefits when used for the controller's MSS (Model Structure Selection) purposes, since the main objective, in the design of a controller, is to make the process output (and not the controller's output), behave as desired.

% Neste sentido, propõe-se adaptações ao método RaMSS original visando objetivos mais adequados a MSS de controladores. Uma modificação estudada, é a substituição da versão exponencial do MSSE, dada por $J_s= e^{-K\cdot MSSE}$, por uma que leve em conta o rastreamento do sistema em malha fechada, um dos principais objetivos do controlador. Este novo índice é definido como
In this sense, it is proposed adaptations to the original RaMSS method aiming at objectives more suitable to MSS of controllers. A studied modification at the moment is the replacement of the exponential version of the MSSE, given by $J_s= e^{-K\cdot MSSE}$, by one that takes into account the closed-loop system tracking, one of the main objectives of the controller. This new index is defined as
\begin{equation}
  J_r=e^{-K\cdot MSTE},
  \label{eq:Jr}
\end{equation}
where $MSTE$, is defined here as the mean squared tracking error, given by
\begin{equation}
  MSTE = \frac{1}{N}\sum_{k=1}^{N} (y_{rm}(k)-y(k))^2.
  \label{eq:MSTE}
\end{equation} 

% O índice de desempenho final $\mathcal{J}$, utilizado em $\mathcal{I}$ (vide Equacão \ref{eq:desMedioReg}) para o cálculo dos RIPs, passa a ser calculado por:
The final performance index $\mathcal{J}$, used in $\mathcal{I}$ (see Equation \ref{eq:desMedioReg}) for the computation of RIPs, is now calculated by:
\begin{equation}
  \mathcal{J} = \alpha \mathcal{J}_r + (1-\alpha)\mathcal{J}_p
\label{eq:Jracss}
\end{equation}
% em que $J_p$ continua sendo o mesmo dado por \eqref{eq:Jp}, uma versão exponencial do MSPE. Este novo índice leva em conta o desempenho do controlador em malha fechada no processo de atualização dos RIPs.
where $J_p$ remains the same given by \eqref{eq:Jp}, an exponential version of MSPE. This new index takes into account the performance of the closed loop controller in the process of updating the RIPs.

% Uma vez que a escolha de estrutura é feita pelo método RaMSS modificado, incorporando elementos da estratégia VRFT, e que o intutito é a seleção de estrutura e parâmetros para controladores, batiza-se, aqui, esta estratégia de \textit{Randomized Controler Structure Selection}, ou RaCSS.
Since the choice of structure is made by the modified RaMSS method, incorporating elements of the VRFT strategy, and the aim is the selection of structure and parameters for controllers, this strategy is named here \textit{Randomized Controler Structure Selection }, or RaCSS.

% Resultados preliminares do uso deste índice são apresentados nesta qualificação. Os Exemplos \ref{ex:52} e \ref{ex:53} apresentam alguns destes resultados e uma análise geral é feita no Capítulo \ref{cap:Concl}. \todo{colocar a referencia correta aqui.}
Preliminary results from the use of this index are presented in next sections. Examples \ref{ex:52} and \ref{ex:53} show some of these results and a general analysis is done in Chapter \ref{cap:Concl}.

% Ressalta-se que os resultados apresentados são preliminares e conclusões concretas ainda não puderam ser apresentadas.
\todo{batizar o método de RaCSS.} 
% No Capítulo \ref{cap:Concl}, propostas para continuidade do trabalho, como inclusão de informações auxiliares ao procedimento, objetivando incorporar características previamente desejadas ou conhecidas para o controlador, assim como possivelmente o uso de estratégias de aprendizado por reforço, no intuito de reduzir o custo computacional devido ao uso de informações da malha fechada do controlador são apresentadas.
In Chapter \ref{cap:Concl}, proposals for continuing the work, such as the inclusion of auxiliary information to the procedure, aiming to incorporate previously desired or known characteristics for the controller, as well as possibly the use of reinforcement learning strategies, in order to reduce the computational cost due to the use of closed loop information from the controller are presented.

% Por fim, discussões a respeito de estabilidade, convergência, polarização e robustez são apresentadas, ainda não em um contexto de garantias, mas como conjecturas de passos que pretende-se aprofundar na sequência da presente pesquisa.
Finally, discussions about stability, convergence, polarization and robustness are presented, not yet in a context of guarantees, but as conjectures of steps that we intend to deepen in the sequence of this research.

\section{Preliminary Results}\label{sec:prel_results}

% -*- TeX-master: "/home/joao/DOUTORADO/GIT/Qualificacao/Qualificacao.tex" -*-
%!TEX root = ../Qualificacao.tex

%%%%%%%%%%%%%%%%%%%%%%%%%%%%%%%%%%%%%%%%%%%%%%%%%%%%%%%%%%%%%%%%%%%%%%%
%                 EXAMPLE 4.1 (51) - 2nd Order System                 %
%%%%%%%%%%%%%%%%%%%%%%%%%%%%%%%%%%%%%%%%%%%%%%%%%%%%%%%%%%%%%%%%%%%%%%%



% exemplo 51
\begin{exmp}[Second order Linear System -- continued.]\label{ex:sis2aord}

  Consider the 2nd-order linear system describe in Example \ref{exm:31}: 
  \begin{equation}
    \label{eq:sis2aord}
    y_k = a_1y_{k-1} + a_2y_{k-2} + b_1u_{k-1} + b_2u_{k-2}.
  \end{equation}
  % onde, para $i = 1,\ 2$, os termos $a_i$, $b_i$ $\in \R$, representam parâmetros constantes; $u_{k-i}$ e $y_{k-i}$ $\in \R$, os sinais de entrada e saída, respectivamente; e $k$ o índice temporal.
  According to the VRFT strategy (Chapter ~\ref{cap:VRFT}), one must first define a class of allowable controllers $\mathscr{C}$, containing the desired  structure, and define a reference model that expresses the desired close loop system's behavior.

  In this example, we want to analyze the use of the RAmSS algorithm directly, without modifications, in order to find the best structure for a controller designed from the VRFT strategy.
  In order to examine the behavior of the RaMSS Algorithm in finding the best set of regressors, we proceed as follows.

  A feasible reference model is defined, that is, with a relative degree equal to or greater than that of the process. For simplicity, a 1st order model with the same relative degree of the process is chosen, as Example \ref{exm:31}, given by the transfer function
  \begin{equation}
    M(z) = \frac{1-A}{z-A},
    \label{eq:mr_sis2aord}
  \end{equation}
  The parameter adopted here is the same as Exemple \ref{exm:31}, i.e. $A = -T_s/\tau_d = 0.8187$. 
  The ideal controller, represented by an ARX model, will be given by the structure composed of the regressors presented in \eqref{eq:exp31_uk}, represented here by
  \begin{equation}
    \label{eq:exp51_contIdeal}
    u_k = \theta_0e_{k} + \theta_1e_{k-1} + \theta_2e_{k-2} + \theta_3u_{k-1} + \theta_4u_{k-2},
  \end{equation}
  and will have the parameters shown in \eqref{eq:exp31_ideal_parameters}, described here as
  \begin{equation}
    \vtheta^\star= \begin{bmatrix} \theta^\star_0 & \theta^\star_1 & \theta^\star_2 & \theta^\star_3 & \theta^\star_4 \end{bmatrix}^T =  \begin{bmatrix} 0.181 & 0.308 &  0.145 &  0.19 & 0.81 \end{bmatrix}^T.
  \label{eq:ex51_ideal_parameters}
\end{equation}
Using the RaMSS algorithm, as discussed in the section ~\ref{sec:ramss}, the procedure is performed for 2 cases:
\begin{description}
  \item[case A] the universe set $\mathscr{M}$ is taken as all possible 3rd degree non linear models formed by the monomials up to 4 delays for input $\tilde{e}_k$ (virtual error) and output $\tilde{u}_k$ (virtual process input) collected data. No noise is considered in this case.
  \item[case B] the same case as case A, but now with a noise in the output, given by a gaussian distribution, i.e. $\nu \sim \mathcal{N}(\mu,\sigma)$, where $\mu=0$ is the mean, and $\sigma= 0.1$ is the standard deviation adopted. Note that the noise is added in the output, in a way that the model for the process is represented by an output error model (OEM).
\end{description}
The RaMSS procedure is applied to a data set of 700 samples, obtained via the VRFT procedure, in which the VRFT filter is considered to be unitary, i.e. the data is not filtered. The same data set is applied for the 2 cases. At each iteration, 100 candidate models are randomly chosen to be analyzed, using a Bernoulli process. The RIPs are updated in a maximum of 100 iterations, or until they all converge to a margin above 0.9 or below 0.1, i.e. the following stopping criterion is adopted:
\begin{align}
  \textbf{if } \left[iter<iter_{\max}\right] \textbf{ or } \left[\Delta_S < \frac{1}{2} \left(1-\min_{\forall \mu \in \bm{\mu}_k}{ \left| 2\mu -1 \right| }\right)\right] \textbf{ then} \text{, STOP the procedure.}
\label{eq:crit.par}
\end{align}
where $\Delta_S = 0.1$ is the convergence margin, $iter$ is the iteration number and $iter_{\max} = 100$ is the maximum allowed iterations.


Applying the procedure in both cases, the following controllers are obtained, for a specific realization:
\begin{align}
  % Para caso 24:
  \label{eq:ex51CasesAB}
  u_1(k) &= 0.636{u}_1(k-1) + 0.513{u}_1(k-2) + -0.321{u}_1(k-3) + 0.17{u}_1(k-4) \nonumber\\
      &\quad+ 0.181{e}_1(k) + 0.227{e}_1(k-1) + 0.045{e}_1(k-2) + 0.03{e}_1(k-4) \\
  u_2(k) &= 0.737{u}_2(k-1) + 0.242{u}_2(k-2) + -0.329{u}_2(k-3) + 0.32{u}_2(k-4) \nonumber\\
         &+ 0.175{e}_2(k) + 0.197{e}_2(k-1) + 0.049{e}_2(k-2) + 0.049{e}_2(k-3) + 0.059{e}_2(k-4)
  % u_1(k) &= 0.193{u}_1(k-1) + 0.805{u}_1(k-2)  \nonumber \\
         % &\quad + 0.001{u}_1(k-3) + 0.181{e}_1(k) + 0.307{e}_1(k-1) + 0.144{e}_1(k-2) \nonumber \\
  % u_2(k) &= 0.737{u}_2(k-1) + 0.242{u}_2(k-2) + -0.329{u}_2(k-3) + 0.32{u}_2(k-4)  \\
         % &\quad + 0.175{e}_2(k) + 0.  97{e}_2(k-1) + 0.049{e}_2(k-2) + 0.049{e}_2(k-3) + 0.059{e}_2(k-4)  \nonumber
  % u_1(k) &= 0.193{u}_1(k-1) + 0.805{u}_1(k-2) + 0.001{u}_1(k-3) \\
         % &+ 0.181{e}_1(k) + 0.307{e}_1(k-1) + 0.1441{e}(k-2), \\
  % u_2(k) &= 0.751{u}_2(k-1) + 0.241{u}_2(k-2) -0.33{u}_2(k-3) + 0.322{u}_2(k-4) \\
         % &+ 0.174{e}_2(k) + 0.197{e}_2(k-1) + 0.048{e}_2(k-2) + 0.049{e}_2(k-3) + 0.0591,
\end{align}
where the index subscribed to the variables represent the respective cases.

The table ~\ref{tab:exp51_param} sumarizes the simulation parameters used in \ref{alg:RaMSS}, for the 2 cases,
where $o$ is the maximum allowed degree for the regressors, $n_{\tilde{e}}$ is the maximum delay for the virtual error signal $\tilde{e}(k)$, $n_{\tilde{u}}$ is the maximum delay for the input signal of the sampled plant, $ N_p$ is the number of models chosen at each update of the RIPs, $ iter_{\max} $ is the maximum number of allowed iterations, $\Delta_S$ is the trashold for convergence of RIPs, $K$ is the gain for the performance indexes presented in \ref{eq:Js}, $\gamma_0$ is the initial gain of \ref{eq:gamma}, $\mu_{\min}$ and $\mu_{\max}$ are the minumum and maximum values allowed for the RIPs and $\nu$ is the noise added to the output.
\begin{table}[htpb]
  \centering
  \caption{Parameters for simulating the RaMSS algorithm of the example ~\ref{ex:sis2aord}}\label{tab:exp51_param}
  \begin{tabular}{c|c|c|c|c|c|c|c|c|c|c}
    Case & $o$ & $n_{\tilde{e}}$ & $n_{\tilde{u}}$ & $ N_p$ & $ iter_{\max} $ & $K$ & $\gamma_0$ &  $\mu_{\min}$ & $\mu_{\max}$ & $\nu$\\
    \hline
    A & $ 3 $ & $4$ & $4$ & $100$ & $100$ & $1$ & $2$ & $0.05$ & $1$ & $0$ \\
    B & $ 3 $ & $4$ & $4$ & $100$ & $100$ & $1$ & $2$ & $0.05$ & $1$ & $\mathcal{N}(\mu,\sigma)$
  \end{tabular}
\end{table}\\
\todo[inline]{MEXER NESTA TABELA AINDA! Valores estão errados. Já os modifiquei. Definir parâmetros, etc. Colocar os mais essenciais por aqui e os menos, no appendice. }

Note that the controllers presented in~\eqref{eq:ex51CasesAB} are obtained for a specific realization of the procedure. All terms that end with the RIP above 0.95 are considered in the hundredth iteration.
Figure \ref{fig:ex51_RIPevol_2cases} shows the evolution of the RIPs for this realization.

\begin{figure}[H]
  \centering
  % \includegraphics[width=\textwidth]{Figs/Cap5/ex51_rips_evol_2cases.tex.pdf}
  \includegraphics[width=\textwidth]{Figs/Cap5/ex51_rips.tex.pdf}
  \caption{Typical evolution of RIPs for choosing regressors for case 1 and 2.}
  \label{fig:ex51_RIPevol_2cases}
\end{figure}
\todo[inline]{Trocar esta figura} 

The ideal regressors are selected before the first 40 steps.
But as the iterations continue, the $\tilde{u}(k-3)$, $\tilde{u}(k-4)$ and $\tilde{e}(k-4)$ regressors continue to increase monotonically in value and eventually end up being selected.
The effect of including these terms, although in general they are small, deteriorates the desired behavior of the controller.
When considering noise in the measurement (case 2), the selection of non-ideal regressors occurs even earlier, as can be seen in the lower graph of \ref{fig:ex51_RIPevol_2cases}.
Part of this result is due to the worsening of parametric identification during the VRFT procedure, provoked by the polarization effect introduced by noise at the output. The result is that the regressor vectors are no longer orthogonal to the residuals and the OLS estimator becomes polarized. A strategy to mitigate this effect is to use non-polarized estimators, such as VI estimators, or even the ELS.

Figure ~\ref{fig:Figs-RespostaSist2aordNARX-png} shows the temporal response for the controllers identified in \eqref{eq:ex51CasesAB}, when the reference signal is taken as a square wave.

% The figure  shows the temporal response for a square wave in reference signal, for the 2 cases.
% Note que no Caso 1,

\begin{figure}[H]
  \sbox0{\blacksolidlinethin} \sbox1{\bluedashedline} \sbox2{\reddottedline} \sbox3{\blackdottedline} \sbox4{\bluedashdotedline} 
  \centering
  % \includegraphics[width=1\textwidth]{./Figs/Cap5/ex51_resp_temporal_mf_editado.tex.pdf}
  \includegraphics[width=1\textwidth]{./Figs/Cap5/ex51_resp_temporal_mf2.tex.pdf}
  % \include{./Figs/Cap5/ex51_resp_temporal_mf.tex}
  % \caption{Resposta do sistema em malha fechada (gráfico superior) e respectivos erros absolutos (gráfico inferior) utilizando os controladores identificados no caso A (\usebox1) e no caso B (\usebox2). Os sinais de referência (\usebox3) e de reposta do modelo de de referência (\usebox0) são mostrados no gráfico superior. O erro para caso considerando somente a estrutura ideal é representado por (\usebox4), no gráfico inferior.}
  \caption{Closed-loop system response (upper graph) and respective absolute errors (lower graph) using the controllers identified in case A (\usebox1) and case B (\usebox2). The reference (\usebox3) and the reference model response (\usebox0)  signals are shown in the upper graph. The error for case considering only the ideal structure is represented by (\usebox4), in the lower graph.}
  \label{fig:Figs-RespostaSist2aordNARX-png}
\end{figure}

The behavior of the case with noise is deteriorated in relation to case 1 (without noise). The steady-state error for case 2 is about 10 times greater than case 1, as shown in the lower graph in Figure \ref{fig:Figs-RespostaSist2aordNARX-png} (note the different scales in the error graph in the figure).
This greater error is due to two factors: selection of an over-parameterized structure of the model by the RaMSS procedure, and worse parametric identification due to the presence of noise at the output, which can result in parameter polarization.
Note that, as the noise is added to the process output (OEM), and by the procedure of filtering by the inverse of the plant when applying the VRFT, the noise, although white, can cause polarization in the identified parameters.
The error graph in Figure \ref{fig:Figs-RespostaSist2aordNARX-png} also shows the time response for the case without noise, but considering that only the ideal regressors are taken into account in the identification process.
In this case, the error decreases, and and this decrease is attributed to the over parameterization caused by the extra regressors selected in case 1.
Despite this, a small error remains on steady state. This fact is attributed to the effect discussed in Chapter \ref{cap:VRFT}, in which small errors in the identification of parameters make the sum of the coefficients in $y(k)$ not exactly zero, resulting in a system with high gain at low frequencies, but not infinite. As discussed, in the chapter \ref{cap:VRFT} and to be proposed in the chapter \ref{cap:Concl}, it is hoped that this problem can be solved by imposing restrictions on the identification process (use of auxiliary information).



\todo[inline]{Ainda em construção, por enquanto só coloquei alguns dos gráficos que irei utilizar. Comparações, com tabelas comparando resultados do RaCSS com o ERR também serão ainda colocados. Logicamente, com devidas análises.}

\end{exmp}






The results discussed in the example \ref{ex:sis2aord}, take into account only one performance of one specific realization. A more realistic analysis would be to consider various realizations, and observe the behavior in the form of statistical distributions. These results are shown in the next example.

%!TEX root = Qualificacao.tex

\begin{exmp}[RaCSS applied to a 2nd order linear system] \label{ex:52}

In this example, the same cases contemplated in the previous example are considered, but with the following modifications:
\begin{enumerate}
   \setlength\itemsep{0.1pt}
   \item The index used to calculate the RIPs \eqref{eq:Jcal}, is formed according to \eqref{eq:Jracss}, i.e., the RaCSS procedure is used, described in Section \ref{sec:CSS_metod}, now taking into account the tracking error information in the RIPs updating procedure.
    \item Several simulations, or realizations, are analyzed, but maintaining the same training data, i.e. $\tilde{\bm{u}}$ and $\tilde{\bm{y}}$ remain the same for each realization. What changes are the candidate models chosen during the procedure, since this choice is based on a Bernoulli process.
\end{enumerate}

As a first step, the evolution of RIPs are analyzed for 5 different values of $\alpha$, which are: $\alpha=0$, $\alpha=0.25$, $\alpha=0.5$, $\alpha=0.75$, $\alpha=1$.

% \begin{description}
   % \setlength\itemsep{0.1pt}
  % \item[caso 1] $\alpha =0$, i.e, $J=J_s$;
  % \item[caso 2] $\alpha = 0.25$, i.e. $J=0.25J_s+0.75J_p$;
  % \item[caso 3] $\alpha = 0.5$, i.e. $J=0.5J_s+0.5J_p$;
  % \item[caso 4] $\alpha = 0.75$, i.e. $J=0.75J_s+0.25J_p$;
  % \item[caso 5] $\alpha = 1$, i.e, $J=J_r$.
% \end{description}
%
The 5 cases, considering that there is no measurement noise and with the same values as Example \ref{ex:sis2aord}, are analyzed for the RaCSS procedure. The evolution of the RIPs for each case is presented by Figure \ref{fig:exp51_ev_rips_a1_SR}.
  \begin{figure}[htpb]
    \centering
    \includegraphics{Figs/Cap5/ex51_rips_evol_SR.tex.pdf}
    \caption{RIPs evolution to different values of parameter $\alpha$ considering data without noise.}
    \label{fig:exp51_ev_rips_a1_SR}
  \end{figure}
It is observed that when increasing the value of $\alpha$, regressors that do not belong to the set of ideal regressors are less frequently selected, when compared to the case where no closed loop information is used (case with $\alpha=0$) \footnote { note that the case where $\alpha=0$ in this example corresponds to case A of Example \ref{ex:sis2aord}.}.

Note that, for this realization, for $\alpha=0.5$ and $\alpha=1$ the algorithm was interrupted around iterations 60 and 90, respectively, due to the stop criterion \eqref{eq:crit.par} having been satisfied. It is emphasized that this occurs for this specific realization, the same may not happen for other realizations, due to the random character of the method.
Despite this, in general, when performing several iterations, it is possible to notice that the general behavior of Figure \ref{fig:exp51_ev_rips_a1_SR} prevails, with greater choices of spurious RIPs for the case, in which $\alpha=0$, when compared to the others ($\alpha>0$).

To analyze the behavior more generally, 50 realizations are made (with the same training data) for each value of $\alpha$. The density of probability of convergence of the RIPs for each value of $\alpha$, as a function of the number of iterations, is shown in Figure \ref{fig:exp51_dens_prob_SR}.
\begin{figure}[htpb]
  \centering
  \includegraphics{Figs/Cap5/ex51_iter_con_SEM_ruido.tex.pdf}
  \caption{Probability densities of selection for the regressors selected within 100 iterations considering different values of $\alpha$ for the case without noise.}
  \label{fig:exp51_dens_prob_SR}
\end{figure}
In the figure, it is clear that the method selects the ideal parameters well for all cases. In the case without using the tracking error information, $\alpha=0$, the $\tilde{u}(k-3)$ regressor is chosen with low probability for few iterations but with increasing probability with the increase in iterations.

When tracking error information is taken into account, cases with $\alpha>0$, only the ideal regressors are selected for all 50 realizations. Note that the overall average time to choose the regressors increases with the increase in the $\alpha$ parameter. One possible explanation is that the performance index related to the tracking error, $J_r$, changes little compared to the index related to the prediction error, $J_p$. A probable solution may be to adopt a larger $K$ gain in the calculation of the MSTE (see equation \ref{eq:Jr}).

% As mesmas simulações anteriores são feitas para o caso em que existe ruído de medição (ruído na saída), como no caso B do Exemplo \ref{ex:sis2aord}. A Figura \ref{fig:exp51_ev_rips_a1_CR} mostra a evolução dos RIPs e a Figura \ref{fig:exp51_dens_prob_CR} mostra a densidade de probabilidade de escolha dos regressores em função do número de iterações para cada valor de $\alpha$, considerando esse novo caso.
The same previous simulations are done for the case in which there is measurement noise (noise at the output), as in case B of Example \ref{ex:sis2aord}. Figure \ref{fig:exp51_ev_rips_a1_CR} shows the evolution of RIPs and Figure \ref{fig:exp51_dens_prob_CR} shows the probability density of choice of regressors as a function of the number of iterations for each value of $\alpha$, considering this new case.

  \begin{figure}[htpb]
    \centering
    \includegraphics{Figs/Cap5/ex51_rips_evol_CR.tex.pdf}
    \caption{RIPs evolution to different values of parameter $\alpha$ considering data with noise.}
    \label{fig:exp51_ev_rips_a1_CR}
  \end{figure}

    % \footnote{note que são mostrados em cores os RIPs mais relevantes, i.e. aqueles que não convergem para $\mu_{\min}$). Estes outros são mostrados em escala de cinza e não aparecem na legenda por serem muitos (220 regressores). Os repressores ideais são mostrados em linhas mais espessas.}


  \begin{figure}[htpb]
    \centering
    \includegraphics{Figs/Cap5/ex51_iter_con_COM_ruido.tex.pdf}
    \caption{Probability densities of selection for the regressors selected within 100 iterations considering different values of $\alpha$ for the case without noise.}
    \label{fig:exp51_dens_prob_CR}
  \end{figure}

  % Observando a Figura \ref{fig:exp51_ev_rips_a1_CR}, observa-se que os seguintes fatores: 1) em todos os casos os regressores ideais foram selecionados. Com o aumento de $\alpha$ os regressores não ideais têm uma menor densidade de probabilidade de serem escolhidos. Isto fica mais evidente comparando-se várias realizações, como mostra a Figura \ref{fig:exp51_ev_rips_a1_CR}. Comparando $\alpha=0$ com $\alpha>0$ nota-se que a probabilidade de ser selecionar regressores não ideiais diminui para menores iterações. Porém para valores maiores de $\alpha$, nota-se um maior tempo de convergência dos regressores. Por exemplo, nota-se uma quantidade maior de regressores que não convergem para o valor mínimo $\mu_{\min}$ para maiores valores de $\alpha$. O problema da polarização pode ser um dos resposponsáveis por este fator.
  Looking at Figure \ref{fig:exp51_ev_rips_a1_CR}, it is observed the following facts: 1) in all cases the ideal regressors were selected. 2) With the increase of $\alpha$, non-ideal regressors have a lower probability of being chosen. This is more evident when comparing several realizations, as shown in Figure \ref{fig:exp51_dens_prob_CR}. Comparing $\alpha=0$ with $\alpha>0$, it can be seen that the probability of selecting non-ideal regressors decreases for smaller iterations. However, for values greater than $\alpha=0$, there is a longer convergence time for the regressors. For example, we notice a greater number of regressors that do not converge to the minimum value $\mu_{\min}$ for higher values of $\alpha$. The polarization problem may be one of the factors responsible for this behavior.

  % \todo[inline]{    Neste exemplo aplico o ``RaCSS'' ao mesmo sistema de 2a ordem do exemplo anterior, mas agora usando informação do erro de rastreamento no cálculo dos RIPs. \\ Por enquanto estão somente os gráficos, ainda falta análise e algumas tabelas comparativas com ERR.  }

\end{exmp}



Nos exemplos anteriores, o algoritmo RaCSS é analisado para um caso em que o processo é linear e o controlador ideal é conhecido. Na sequência, apresenta-se um exemplo de aplicação para um processo não linear.


%!TEX root = Qualificacao.tex

\begin{exmp}[RaCSS aplicado a um sistema não-linear] \label{ex:53}

  \todo[inline]{
    Neste exemplo a ideia é avaliar o comportamento do ``RaCSS'' a um sistema não-linear. Tomei um modelo de um aquecedor com dissipação variável adotado no seu livro. A ideia é apresentado o modelo no exemplo~\ref{ex:varHeater} da seção~\ref{sec:ramss}, no sentido do RaMSS para identificar o modelo. Depois aproveito o mesmo aqui para identificar o controlador.

    Por enquanto estão somente os gráficos, ainda falta análise e algumas tabelas comparativas com ERR.
  }

  % Neste exemplo o comportamento do algoritmo RaCSS é analisado na seleção de estrutura para um sistema não linear. É adotado o modelo de um pequeno aquecedor elétrico, com dissipação variável. A variação da dissipação é resultado do acionamento de um ventilador. O sinal de entrada é a tensão elétrica aplicada ao aquecedor e a saída é o sinal amplificado de um termopar. O modelo obtido por um processo de identificação em um sistema real, conforme detalhado em \citep{aguirre2015}, seção 16.6., é dado por
  In this example, the behavior of the RaCSS algorithm is analyzed in the structure selection for a non-linear process. The model of a small electric heater, with variable dissipation, is adopted. The variation in dissipation is the result of the activation of a fan. The input signal is the electrical voltage applied to the heater and the output is the amplified signal from a thermocouple. The model obtained by an identification process in a real system, as detailed in \citep{aguirre2015}, section 16.6., Is given by
 \begin{align}
   y(k) &= 1.2929y(k-1) + 0.0101u(k-2)u(k-1) + 0.0407u^2(k-1) - 0.3779y(k-2) \\
        &\quad - 0.1280u(k-2)y(k-1) + 0.0957u(k-2)y(k-2) + 0.0051u^2(k-2)
 \label{eq:ex53_model}
 \end{align}
 
 % Os são utilizados os mesmos sinais amostrados em \ref{aguirre2015}, conforme apresentados na Figura \ref{fig:ex53_samp_data}.
 The same signals sampled presented in \ref{aguirre2015} are used to calculate the virtual values used in the RaCSS procedure. These signals are shown in Figure \ref{fig:ex53_samp_data}.
\begin{figure}[htpb]
  \centering
  \includegraphics[width=\textwidth]{Figs/Cap5/ex53_sample_data.tex.pdf}
  \caption{Entrada e saída do aquecedor, em p.u.}
  \label{fig:ex53_samp_data}
\end{figure}
 % O período de amostragem original é de 12 segundos, porém, por simplicidade, será considerado aqui que este período de amostragem é de um segundo, $T_s =1$ s. Isto não afeta em nada os resultados a serem obtidos para fins deste exemplo numérico. Fazendo esta simplificação considera-se um modelo de referência dado por um sistema de primeira ordem, com atraso de transporte de 3 segundos e com constante de tempo de 10 segundos, resultando na seguinte função de transferência no tempo contínuo
The original sampling period is 12 seconds, however, for simplicity, it will be considered here that this sampling period is one second, $T_s =1$ s. This has no effect on the results to be obtained for the purposes of this numerical example. Making this simplification, it is considered a reference model given by a first order system, with a transport delay of 3 seconds and a time constant of 10 seconds, resulting in the following transfer function in continuous time
\begin{equation}
   M(s) = \frac{e^{-\tau_d s}}{\tau s-1},
\label{eq:}
\end{equation}
% em que $\tau = 10$ é a constante de tempo desejada e $\tau_d = 3$ é o atraso de transporte. A versão discreta deste modelo pode ser escrita como:
where $\tau = 10$ s is the desired time constant and $\tau_d = 3$ s is the transport delay. A discrete version of this model is given by
 \begin{equation}
   M(z) = \frac{1-e^{-T_s/\tau}}{(z-e^{-Ts/\tau})z^{\tau_d/Ts-1}}.
   \label{eq:ex53_MRcont}
 \end{equation}

 % Desconsiderando ruídos no processo, o procedimento RaCSS é aplicado afim de se obter um controlador que resulte em um sistema em malha fechada que se comporte de acordo com um modelo de referência dado.
 Disregarding noise in the process, the RaCSS procedure is applied in order to obtain a controller that results in a closed-loop system that behaves according to a given reference model.
 % Considera-se como regressores candidatos todos os regressores formados por monômios até 8 atrasos em $\tilde{u}$ (sinal de saída do controlador) e em $\tilde{e}$ (erro virtual).
 It is considered all regressors formed by monomials up to 8 delays in $\tilde{u}$ (controller output signal) and $\tilde{e}$ (virtual error) as candidate regressors to RaCSS.
 % Os parâmetros adotados no algorítmo RaCSS são apresentados na \ref{tab:exp53_param}, onde $\alpha$ é o fator de acoplamento dado em \eqref{eq:Jracss} e os significados dos outros símbolos são os mesmos da \ref{tab:exp51_param}.
 The parameters adopted in the RaCSS algorithm are presented in Table \ref{tab:exp53_param}, where xxxxxxxx is the factor given in \eqref{eq:Jracss} and the meanings of the other symbols are the same as presented in Table \ref{tab:exp51_param} .

\begin{table}[htpb]
  \centering
  \caption{Parameters used in the RaMSS algorithm of the example ~\ref{ex:53}}\label{tab:exp53_param}
  \begin{tabular}{c|c|c|c|c|c|c|c|c|c|c}
    $o$ & $n_{\tilde{e}}$ & $n_{\tilde{u}}$ & $ N_p$ & $ iter_{\max} $ & $K$ & $\gamma_0$ &  $\mu_{\min}$ & $\mu_{\max}$ & $\nu$ & $a$\\
    \hline
     $ 2 $ & $8$ & $8$ & $100$ & $200$ & $1$ & $2$ & $0.01$ & $1$ & $0$ & $0.5$ \\
  \end{tabular}
\end{table}

% Uma evolução típica para os RIPs, após 200 iterações é mostrada na Figura \ref{fig:ex53_rips}.
A typical evolution for the RIPs, after 200 iterations is shown in Figure \ref{fig:ex53_rips}.

  \begin{figure}[htpb]
    \centering
    \includegraphics[width=\textwidth]{Figs/Cap5/ex53_RIPs.tex.pdf}
    \caption{RIPs evolution for example \ref{ex:53}.}
    \label{fig:ex53_rips}
  \end{figure}

  % Nota-se um comportamento bem menos regular com aqueles obtidos no Exemplo \ref{ex:52}. Ao final das 200 iterações, 15 regressores são selecionados. Apesar de ainda existirem outros regressores que ainda não convergiram para o valor mínimo e máximo, um certo grau de classificação já pode ser obtido a partir dos resultados. Ressalta-se que são um total de 171 possíveis regressores, o que resulta em um total de, aproximadamente, $2^{171}$ modelos possíveis.
  There is a much less regular behavior with those obtained in Example \ref{ex:52}. At the end of the 200 iterations, 15 regressors are selected. Although there are still other regressors that have not yet converged to the minimum and maximum value, and a certain degree of classification can already be obtained from the results. It should be noted that there are a total of 171 possible regressors, which results in a total of approximately $2^{171}$ possible models.

  % A resposta em malha fechada considerando-se os primeiros 15, 19 e 27 regressores é mostrada na Figura \ref{fig:ex31_resp_temporal}.
  The closed loop response considering the first 15, 19 and 27 regressors is shown in Figure \ref{fig:ex31_resp_temporal}.
  The MAPE, MSTE and $J_r$ index values (see \eqref{eq:MSTE} and \eqref{eq:Jracss}) are presented in Table \ref{tab: exp53_mape}.

  \begin{figure}[htpb]
  \sbox0{\blacksolidlinethin} \sbox1{\bluedashedline} \sbox2{\greendashdottedline} \sbox3{\reddashedline} \sbox4{\blackdottedline} 
    \centering
    \includegraphics[width=\textwidth]{Figs/Cap5/ex53_resp_temp.tex.pdf}
    \caption{Closed loop response for controllers obtained using the first 15 (\usebox1), 19 (\usebox2) and 27 (\usebox3) RIPs obtained using RaCSS, together with the reference signal (\usebox4) and the reference model response (\usebox0).}
    \label{fig:ex31_resp_temporal}
  \end{figure}

  % Os valores do MAPE, do MSTE e do índice $J_r$ (vide \eqref{eq:MSTE} e \eqref{eq:Jracss}) são apresentados na Tabela \ref{tab:exp53_mape}.


  % Observa-se que a ao se considerar 27 regressores a resposta fica praticamente idêntica a desejada, com um valor MAPE de apenas 2.14. Nenhuma análise posterior foi feita sobre os parametros classificados. Em uma nova análise pretende-se separar os regressores selecionados e proceder novamente com o procedimento RaCSS, mas desta vez, somente com estes regressores já pré-selecionados.
  It is observed that when considering 27 regressors the answer is practically identical to the desired one, with a MAPE value of only 2.14. No further analysis was done on the classified parameters. In a new analysis, we intend to separate the selected regressors and proceed again with the RaCSS procedure, but this time, only with these pre-selected regressors.

\begin{table}[htpb]
  \centering
  \caption{Parameters used in the RaMSS algorithm of the example ~\ref{ex:53}}\label{tab:exp53_mape}
  \begin{tabular}{c|c|c|c}
     & $MAPE$ & $MSTE$ & $J_s$ \\
    \hline
    15 & 69.02 & 0.2142 & 0.8072 \\
    19 & 9.59 & 0.0050 & 0.9950 \\
    27 & 2.14 & 0.00116 & 0.9988 \\
  \end{tabular}
\end{table}


\end{exmp}





% {{{ Anotacoes

% \newpage
% \section{Rascunho}\label{sec:Rascunho}
%
%
% Como produto/trabalho em desenvolvimento durante o semestre, cita-se o trabalho que tem sido feito no sentido de desenvolver metodologias para escolha de estruturas para o controlador no âmbito do controle baseado em dados, ou DDC (Data-Driven Control).
% Mais especificamente, tem-se dado ênfase a escolha da estrutura do controlador a partir do uso de métodos VRFT.
% Como deixa claro em seu nome, ao usar o termo “sintonia” (em inglês, “tunning” ), a metodologia VR  FT visa a sintonia de um controlador pertencente a uma classe de controladores pré-estabelecida.
% Como abordado pelos autores do método~\citep{campi2002,campi2006a}, por outros estudiosos do assunto~\citep{bazanella2012} e apresentado neste relatório na seção de revisão bibliográfica, o método visa minimizar o erro de rastreamento de um modelo de referência pré-definido a partir da minimização de uma função de custo.
% Tal função é definida a partir do erro quadrático entre o sinal de excitação utilizado no processo a se controlar e o sinal de controle “virtual”, gerado pelo controlador identificado ao se aplicar  o mesmo sinal de referência capaz de produzir o mesmo sinal de saída  virtual utilizado na identificação.
%~\cite{campi2002} mostram, para o caso linear, e~\cite{campi2006}  e não linear, que ao se minimizar esta nova função de custos, minimiza-se também o erro de rastreamento, desde que se obedeça os seguintes critérios: a classe do controlador considerado tenha uma estrutura que possibilite abrigar o que se nomeia de controlador ideal (vide eq.
% ); que o processo não seja afetado por ruídos; e o controlador seja parametrizado linearmente.
% Caso o controlador ideal não possa ser representado pela classe considerada, o erro mínimo de rastreamento não é mais garantido e a introdução de um filtro ao projeto, visando se aproximar desse mínimo é desejado.
%
%
% No desenvolvimento do trabalho, foco deste relatório, tem-se feito um estudo de como seria possível, e quais as vantagens poderia-se obter se, ao invés de apenas tentar achar a melhor sintonia para um controlador com estrutura pré-estabelecida,  se procurasse também uma melhor estrutura dentro de um um conjunto de classes de controladores, de forma que o erro de rastreamento ótimo ou pelo menos sub-ótimo possa ser encontrado.
%
% Para atingir este propósito, tem-se investigado o uso de técnicas de seleção de estruturas que, com devidas adaptações e reformulações, sejam capazes selecionar modelos que resultem em controladores não lineares (ou mesmo lineares) que possam levar a respostas melhores do que aqueles que fazem uso de estruturas pré definidas.
%
% Dentre as ferramentas que têm-se utilizado, destacam-se o uso da taxa de redução de erro (ERR) e do algoritmo   Randomized Model Structure Selection (RaMSS)~\citep{falsone2014,falsone2015}. Estas estratégias têm sido estendidas e utilizadas em aplicações para determinação de modelos de processos dinâmicos~\citep{retesNARMAXModelIdentification2019} ou até mesmo no projeto de compensadores para aplicações em que o processo exibe determinados efeitos de não linearidade específicas, como no caso de efeitos histeréticos~\citep{abreuIdentificationNonlinearityCompensation2020}.

% Em particular, o método RaMSS, tem se mostrado promissor para a aplicação desejada. Esse método faz apelo às técnicas exploratórias que recorrem a buscas aleatórias ao estilo Monte Carlo, mas com mecanismos de seleção que reduzem drasticamente o custo computacional, evitando uma busca exaustiva, ao mesmo tempo em que tenta garantir uma seleção adequada.
%
% Apoiado nesta técnica, algumas adaptações têm sido estudadas no sentido de lidar agora com a identificação do controlador, e não mais do processo.
%
% Como abordado na revisão bibliográfica deste relatório, o método RaMSS faz uso de um índice de desempenho baseado no cálculo de alguma grandeza que quantifique a qualidade de o modelo em algum sentido. Uma escolha comum é calcular este índice baseado no erro quadrático médio de predição (MSPE).
%
% Um estudo sobre o uso deste índice tem sido feito na pesquisa em questão, conforme resultados apresentados na seção “Análise e Processamento de Dados”. O que se tem observado é que minimizar o MSPE diretamete pela estratégia RaMSS apesar de muitas vezes apresentar bons resultados, não dá garantias que o rastreamento ótimo é alcançado, i.e. aquele em que o erro de rastreamento médio quadrático por exemplo, é mínimo.
%
% Para sanar  esta deficiência, tem-se considerado o uso de algum índice que leve em conta o resultado final ao se aplicar os controladores intermediários identificados e selecionados pelo procedimento. Algo parecido tem sido usado no que diz respeito a identificação de processos  em que o erro quadrático médio de simulação (MSSE) é utilizado em substituição ao MSPE~\citep{aguirre2010}. Neste caso, segundo~\citep{piroddi2003}, o uso de informações da simulação-livre pode melhorar a robustez na seleção de estrutura do processo quando sob condições de identificabilidade parciais.
%
% O MSSE depende da simulação livre, que em suma é a resposta em malha aberta a um sinal de excitação conhecido. Algo parecido poderia ser utilizado ao se avaliar a estrutura no procedimento VRFT, porém, neste caso, seria desejável a resposta do sistema em malha fechada com o controlador obtido a partir da estrutura avaliada. O grande problema neste caso é que esta simulação passa a ser dependente de um modelo, ainda que aproximado do processo. Porém, como estratégias DDC visam exatamente não identificar um modelo para o processo, tal simulação torna-se inviável.
%
% No atual estágio desta pesquisa, tem-se trabalhado com a ideia de um índice que quantifique o erro médio de rastreamento em função do modelo do controlador sintonizado por uma estratégia VRFT, de modo que alguma informação da resposta em malha fechada com o controlador analisado possa ser usada para  o cálculo dos índices de probabilidades de inclusão do regressor, ou RIP (vide seção de Revisão Bibliográfica). Porém, como a simulação da resposta em malha fechada (ou mesmo malha aberta) é inviável devido a uma falta de modelo para o processo, o que se estuda é o uso de técnicas de Aprendizado por Reforço (RL) para que dados colhidos do processo real, enquanto em funcionamento, possam ser usados para o cálculo em tempo real dos RIP e consequentemente para  escolha de melhor estrutura.
%
% Algumas técnicas de RL se mostram promissoras neste caso, uma vez que muitas vezes levam à otimização de índices de desempenho a partir de dados amostrados, sem a necessidade do modelo do processo, ao mesmo tempo em que evitam o alto número de realizações amostrais como em metodologias Monte Carlo. Dentre elas a estratégia TD Learning, tem sido considerada, com uma atenção ao método conhecido como Q-learning, que permite que um controlador possa ser ajustado em uma abordagem conhecida como off-policy. Nesse caso, um controlador ou lei de controle (no caso do presente trabalho, mais precisamente sua estrutura) pode ser determinado enquanto o sistema em malha fechada obedece a uma lei (ou política) de controle diferente. Isto ajuda em eventuais problemas de instabilidade e exploração.
%
% Durante o semestre tem sido desenvolvido um algoritmo no intuito de implementar as ideiais consideradas anteriormente. Parte do algoritmo desenvolvido por Retes (2018), tem sido aproveitada, assim como algoritmos implementados e desenvolvidos pelo grupo do MACSIN1, grupo do qual o autor faz parte desde o início do doutorado. O algoritmo tem sido desenvolvido na linguagem R, portanto traduções de algumas ferramentas já disponíveis no MACSIN em outras linguagens tiveram e estão tendo que ser adaptadas. Os resultados apresentados na seção “Análise e Processamento de Dados” foram obtidos, em sua maioria, a partir deste algoritmo.


%  }}}
