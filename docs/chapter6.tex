% -*- TeX-master: "Qualificacao.tex" -*-
%!TEX root = Qualificacao.tex

\chapter{Conclusions}\label{cap:Concl}
\vspace{-1cm}

% Frase citação inicial {{{1
% \begin{flushright}
% \begin{minipage}{0.7\linewidth}
%     \emph{``\dots''}
% \end{minipage}
% \end{flushright}
%
% \begin{flushright}
% Cicrano
% \end{flushright}
%
%\vspace{1cm}



\section{Final Considerations}\label{sec:Cons_finais}

% A atual pesquisa se encontra em fase de desenvolvimento, mas com os primeiros resultados já é possível inferir que a abordagem RaCSS funciona, no sentido em que auxilia na seleção de estruturas e projeto de controladores DDC. Mais análises serão feitas a fim de se obter informações a respeito de robustez, desempenho e limitações do método.
The current research is in the development phase, but with the first results it is already possible to infer that the RaCSS approach works, in the sense that it helps in the selection of structures and design of DDC controllers. Further analysis will be carried out in order to obtain information regarding the method's robustness, performance and limitations.

% Durante o desenvolvimento, algumas propostas para investigação foram surgindo, as quais, em sua maioria ainda não puderam ser investigadas. Como este trabalho se encontra ainda em fase de desenvolvimento, tais propostas são apresentadas como propostas de continuidade e são listadas com mais detalhes na seção seguinte.
During the development, some proposals for investigation were emerging, which, for the most part, could not be investigated until now. As this work is still in the development phase, these proposals are presented as continuity proposals and are listed in more detail in the following section.

% Como produto/trabalho em desenvolvimento durante o semestre, cita-se o trabalho que tem sido feito no sentido de desenvolver metodologias para escolha de estruturas para o controlador no âmbito do controle baseado em dados, ou DDC (Data-Driven Control).
% Mais especificamente, tem-se dado ênfase a escolha da estrutura do controlador a partir do uso de métodos VRFT.
% Como deixa claro em seu nome, ao usar o termo “sintonia” (em inglês, “tunning” ), a metodologia VR  FT visa a sintonia de um controlador pertencente a uma classe de controladores pré-estabelecida.
% Como abordado pelos autores do método \citep{campi2002,campi2006a}, por outros estudiosos do assunto \citep{bazanella2012} e apresentado neste relatório na seção de revisão bibliográfica, o método visa minimizar o erro de rastreamento de um modelo de referência pré-definido a partir da minimização de uma função de custo.
% Tal função é definida a partir do erro quadrático entre o sinal de excitação utilizado no processo a se controlar e o sinal de controle “virtual”, gerado pelo controlador identificado ao se aplicar  o mesmo sinal de referência capaz de produzir o mesmo sinal de saída  virtual utilizado na identificação.
% \cite{campi2002} mostram, para o caso linear, e \cite{campi2006}  e não linear, que ao se minimizar esta nova função de custos, minimiza-se também o erro de rastreamento, desde que se obedeça os seguintes critérios: a classe do controlador considerado tenha uma estrutura que possibilite abrigar o que se nomeia de controlador ideal (vide eq.
% ); que o processo não seja afetado por ruídos; e o controlador seja parametrizado linearmente.
% Caso o controlador ideal não possa ser representado pela classe considerada, o erro mínimo de rastreamento não é mais garantido e a introdução de um filtro ao projeto, visando se aproximar desse mínimo é desejado.
%
%
% No desenvolvimento do trabalho, foco deste relatório, tem-se feito um estudo de como seria possível, e quais as vantagens poderia-se obter se, ao invés de apenas tentar achar a melhor sintonia para um controlador com estrutura pré-estabelecida,  se procurasse também uma melhor estrutura dentro de um um conjunto de classes de controladores, de forma que o erro de rastreamento ótimo ou pelo menos sub-ótimo possa ser encontrado.
%
% Para atingir este propósito, tem-se investigado o uso de técnicas de seleção de estruturas que, com devidas adaptações e reformulações, sejam capazes selecionar modelos que resultem em controladores não lineares (ou mesmo lineares) que possam levar a respostas melhores do que aqueles que fazem uso de estruturas pré definidas.
%
% Dentre as ferramentas que têm-se utilizado, destacam-se o uso da taxa de redução de erro (ERR) e do algoritmo   Randomized Model Structure Selection (RaMSS) \citep{falsone2014,falsone2015}. Estas estratégias têm sido estendidas e utilizadas em aplicações para determinação de modelos de processos dinâmicos \citep{retesNARMAXModelIdentification2019} ou até mesmo no projeto de compensadores para aplicações em que o processo exibe determinados efeitos de não linearidade específicas, como no caso de efeitos histeréticos \citep{abreuIdentificationNonlinearityCompensation2020}.


\section{Continuity proposals}\label{sec:Prop_cont}

% Como produto/trabalho em desenvolvimento durante o semestre, cita-se o trabalho que tem sido feito no sentido de desenvolver metodologias para escolha de estruturas para o controlador no âmbito do controle baseado em dados, ou DDC (Data-Driven Control).

% O principal foco desta pesquisa tem sido o estudo e desenvolvimento de metodologias de projeto de controladores DDC, com foco especial na seleção de estruturas do controlador com o uso da técnica VRFT.
The main focus of this research has been the study and development of design methodologies for DDC controllers, with a special attention on the controller structures selection using the VRFT approach.

% Em particular, o método RaMSS, tem se mostrado promissor para a aplicação desejada. Esse método faz apelo às técnicas exploratórias que recorrem a buscas aleatórias ao estilo Monte Carlo, mas com mecanismos de seleção que reduzem drasticamente o custo computacional, evitando uma busca exaustiva, ao mesmo tempo em que tenta garantir uma seleção adequada de modelos.
In particular, the RaMSS method, has shown to be promising for the desired application. This method appeals to exploratory techniques that uses Monte Carlo-style random searches, but with selection mechanisms that dramatically reduce computational cost, avoiding an exhaustive search, while trying to ensure an adequate selection of regressors.
%
% Apoiado nesta técnica, algumas adaptações têm sido estudadas no sentido de lidar com a identificação e seleção de estrutura do controlador e não mais do processo.
Supported by this technique, some adaptations have been studied in order to deal with the identification and structure selection of the controller, and not of the process, as proposed by RaMSS approach.

% Na sequência apresenta-se itens específicos a serem abordados, que podem ser tomados como propostas de continuidade:
Following, specific items to be addressed are presented, which can be taken as proposals for continuity of the current work:


% Uma primeira proposta diz respeito ao estudo do uso de índices de desempenho que melhor representem a realidade do controlador ao se atualizar os termos de RIPs.

\medskip
\textbf{Proposal 1} 

A first proposal concerns the study of the use of performance indices that best represent the reality of the controller when updating the terms of RIPs.
% Como já abordado, originalmente o RAmSS utiliza medidas como o MSPE e MSSE para esta atualização.
As already discussed (Section \ref{sec:ramss}), the RaMSS originally uses measures such as MSPE and MSSE for the RIPs updates.
% Para fins de modelagem  de processos físicos, onde muitas vezes a predição do comportamento entrada-saída é o alvo, tais índices se mostram adequados \cite{falsone2015}.
For modeling of physical processes purposes, where the prediction of input-output behavior is often the target, such indices are adequate \cite{falsone2015}.
%
% Porém, ao se abordar o problema da idendificação do controlador, a redução do erro de predição, seja ele de passo a frente, como no caso do MSPE ou de simulação livre, como no caso do do MSSE,  nao ha garantia de que o erro de rastreamento seja reduzido.. E em geral, é este erro de rastreamento o principal alvo em sistemas de controle.
However, when dealing with the controller identification problem, the minimization of the prediction error, be it a step-ahead error, as in the case of MSPE or a free simulation error, as in the case of MSSE, there is no guarantee that the error of tracking is reduced. And in general, this tracking error is the main target in control systems.
%
% omo abordado na revisão bibliográfica deste relatório, o método RaMSS faz uso de um índice de desempenho baseado no cálculo de alguma grandeza que quantifique a qualidade de o modelo em algum sentido. Uma escolha comum é calcular este índice baseado no erro quadrático médio de predição (MSPE).
%
% Um estudo sobre o uso deste índice está em andamento\todo{conforme resultados apresentados na seção ???}. O que se tem observado é que minimizar o MSPE diretamete pela estratégia RaMSS apesar de muitas vezes apresentar bons resultados\todo{não esquecer de explicar o porque destes resultados mesmo quando minimizando o MSPE, que no caso de controle fica em função dos sinais de controle e não da saída. Creio que dá para explicar isso no contexto do VRFT, em que minimizar este índice implica em minimizar o MSE de rastreamento sob certas condições.}, não dá garantias que o rastreamento ótimo é alcançado, i.e. aquele em que o erro de rastreamento médio quadrático é mínimo.
% Um estudo sobre o uso deste índice está em andamento, conforme resultados apresentados no Capítulo \ref{cap:CCS}. O que se tem observado é que minimizar o MSPE diretamete pela estratégia RaMSS apesar de muitas vezes apresentar bons resultados
% não dá garantias que o rastreamento ótimo é alcançado, i.e. aquele em que o erro de rastreamento médio quadrático é mínimo.
\todo{não esquecer de explicar o porque destes resultados mesmo quando minimizando o MSPE, que no caso de controle fica em função dos sinais de controle e não da saída. Creio que dá para explicar isso no contexto do VRFT, em que minimizar este índice implica em minimizar o MSE de rastreamento sob certas condições.}, 
A study on the use of this index is underway, according to results presented in Chapter \ref{cap:CCS}. What has been observed is that minimizing the MSPE directly by the RaMSS strategy, despite often showing good results at prediction of the controller output view point, does not guarantee that the optimal tracking is achieved, i.e. the one in which the mean squared tracking error is minimal.

% Como alternativa, propõe-se o uso de algum índice que leve em conta o resultado em malha fechada ao se aplicar os controladores intermediários identificados e selecionados pelo procedimento. Algo parecido tem sido usado no que diz respeito a identificação de processos  em que o erro quadrático médio de simulação (MSSE) é utilizado em substituição ao MSPE \citep{aguirre2010}. Neste caso, segundo \citep{piroddi2003}, o uso de informações da simulação-livre pode melhorar a robustez na seleção de estrutura do processo quando sob condições de identificabilidade parciais.
As an alternative, it is proposed to use an index that takes into account the result in closed loop when applying the intermediate controllers identified and selected by the procedure. Something similar has been used with regard to the identification of processes in which the mean quadratic simulation error (MSSE) is used to replace the MSPE \citep{aguirre2010}. In this case, according to \citep {piroddi2003}, the use of information from the free simulation can improve the robustness in the selection of the process structure when under partial identifiability conditions.

% O MSSE depende da simulação livre, que em suma é a resposta em malha aberta a um sinal de excitação conhecido. Algo parecido poderia ser utilizado ao se avaliar a estrutura no procedimento VRFT, porém, neste caso, seria desejável a resposta do sistema em malha fechada com o controlador obtido a partir da estrutura avaliada. O grande problema neste caso é que esta simulação passa a ser dependente de um modelo, ainda que aproximado, do processo, ou do próprio processo. Porém, como estratégias DDC visam exatamente não identificar um modelo para o processo, tal situação pode ser um empecilho prático. Uma proposta de solução para este problema é apresentada mais a frente neste texto.
The MSSE depends on free-run simulation, which in short is the open loop response to a known (and new) excitation signal. Something similar could be used when evaluating the structure in the VRFT procedure. In this case, it would be desirable to use the system closed loop response with the controller obtained from the evaluated structure to calculate a new index. As discussed in Chapter \ref{cap:CCS}, the MSE of the tracking error seems to be a good choice (see equation \ref{eq:MSTE}). The big problem in this case is that this simulation becomes dependent on a model, even if approximate, on the process, or on the process itself. However, as DDC strategies aim at exactly not identifying a model for the process, such a situation can be a practical obstacle. A proposal for a solution to this problem is presented later in this text.

\medskip
\textbf{Proposal 2} 

% Uma segunda proposta, a qual também vem sendo analisada, consiste no uso de informações auxiliares durante o processo de identificação dos parâmetros do controlador.
A second proposal, which has also been analyzed, consists in the use of auxiliary information during the process of identifying the parameters of the controller.
% Apesar de já terem sido desenvolvidas técnicas para incorporar informação auxiliar no processo de identificação, por exemplo via restrições e otimização multiobjetivo \citep{barroso2006}, todas estas restrições dizem respeito à planta.
Although techniques have already been developed to incorporate auxiliary information in the identification process, for example via restrictions and multiobjective optimization \citep{barroso2006}, all these restrictions are related to the plant.
% Neste sentido surgem questões como: de que forma estas técnicas podem ser usadas na abordagem DDC?
In this sense, questions arise such as: how can these techniques be used in the DDC approach?
% Seria possível encontrar um análogo da informação auxiliar, usada em métodos tradicionais, para estratégias DDC, em que não há informação da planta?
Would it be possible to find an analogue of auxiliary information, used in traditional methods, for DDC strategies, in which there is no plant information?
% Poderia esta ser definida, por exemplo, a partir restrições que garantam aspectos relevantes ao controle, como limitações de ganho devido a saturação de atuadores, inserção de integradores no controlador, além de aspectos relativos a robustez? Neste sentido, um procedimento para garantir a presença de integradores no modelo do controlador já vem sendo estudado, mas ainda sem resultados conclusivos.
Could this auxiliar information be defined, for example, from restrictions that guarantee aspects relevant to control, such as gain limitations due to saturation of actuators, insertion of integrators in the controller, in addition to aspects related to robustness? In this sense, a procedure to guarantee the presence of integrators in the controller model has already been studied, but still with no conclusive results.

\medskip
\textbf{Proposal 3} 

% Outra proposta a ser estudada diz respeito ao estudo do uso de filtros no processo de identificação do controlador, como é comum na estratégia VRFT. Na estratégia, quando o controlador ideal, ou compatível, não pertence à classe de controladores considerada, ou seja, a hipótese \ref{ass:machedControl} não é satisfeita, um filtro a ser aplicado ao sinal usado no processo de identificação deve ser projetado a fim de se aproximar de uma solução ótima, conforme apresentado por \cite{campi2002,campi2006} e discutido no Capítulo \ref{cap:VRFT}.
Another proposal to be studied concerns in the use of filters in the process of identifying the controller, as is common in the VRFT strategy. In this strategy, when the ideal or compatible controller does not belong to the class of controllers considered, that is, the hypothesis \ref{ass:machedControl} is not satisfied, a filter to be applied to the signals and regressors used in the identification process must be designed in order to approach an optimal solution, as presented by \cite{campi2002, campi2006} and discussed in Chapter \ref{cap:VRFT}.
% Até agora, os resultados apresentados no Capítulo \ref{cap:CCS} não fizeram uso deste filtro. Este deve ser o próximo passo a ser adotado no procedimento em desenvolvimento.
So far, the results presented in Chapter \ref{cap:CCS} have not made use of this filter. This should be the next step to be taken in the procedure under development.
\todo{terminar este pensamento} 

\medskip
\textbf{Proposal 4}

% Um problema inerente tanto a metodologias de identificação de sistemas quanto ao método VRFT é a polarização de parâmetros que aparecem quando o processo a se identificar (seja ele a planta ou controlador) está sujeito a ruído colorido. É comum o uso de estimadores de variáveis intrumentais (VI) para amenizar os efeitos da polarização na abordagem VRFT. No trabalho em desenvolvimento propõe-se estudar o uso do estimador de mínimos quadrados estendido (ELS) para este fim.
A inherent problem to both, systems identification methodologies, and the VRFT method, is the polarization of parameters that appear when the process to be identified (be it the plant or controller) is subject to colored noise. It is common to use instrumental variable estimators (VI) to mitigate the effects of polarization in the VRFT approach. It is proposed to study the use of the extended least squares estimator (ELS) during the controller structure selection and parameter identification, aiming to reduce the polarization.


\medskip
\textbf{Proposal 5} 

% Um problema conhecido em projetos de controladores baseados em modelos de referência, é a escolha do modelo de referência adequado. Principalmente no que diz respeito à escolha do atraso de tempo puro a ser considerado no modelo de referência. Caso seja escolhido um modelo com atrasos menores que os da planta, por exemplo, a identificação do controlador fica fortemente prejudicada. Neste sentido, pretende-se analisar a possibilidade de buscar por uma constante de tempo adequada para o modelo de referência de forma aleatorizada, durante o procedimento de seleção de estruturas.
A known problem in controller designs based on reference models is the choice of the appropriate reference model. Especially with regard to the choice of the pure time delay to be considered in the reference model. If a model is chosen with shorter delays than the plant, for example, the identification of the controller is severely impaired. In this sense, we intend to analyze the possibility of searching for a suitable time constant for the reference model in a randomized way, during the structure selection procedure.
% Outra possibilidade é analisar eventuais estruturas pré estabelecidas para o modelo de referência a fim de possibilitar melhores ajustes do controlador. Por exemplo, durante o procedimento RaCSS, faz-se também uma busca por modelos de referência ordens distintas, mas que apresentem, por exemplo, mesmo tempo de acomodação.
Another possibility is to analyze some pre-established structures for the reference model in order to allow better adjustments of the controller. For example, during the RaCSS procedure, there is also a search for reference models in different orders, but with, for example, the same accommodation time.

\medskip
\textbf{Proposal 6} 

% Uma quarta proposta diz respeito ao problema levantado anteriormente, onde o sinal de saída, e consequentemente o erro de rastreamento são requeridos para o cálculo do índice de atualização dos RIPs no procedimento RaMSS.
One last proposal concerns the problem raised earlier, in first proposal, where the output signal, and consequently the tracking error, are required to calculate the update rate of the RIPs in the RaMSS (or RaCSS) procedure.
% Como a simulação da resposta em malha fechada (ou mesmo malha aberta) é inviável em estratégia DDC, devido a uma falta de modelo para o processo, pretende-se fazer do uso de técnicas de Aprendizado por Reforço (RL), abordadas no Capítulo \ref{cap:RL}, para que dados colhidos do processo real, enquanto em funcionamento, possam ser usados para o cálculo em tempo real dos RIP e consequentemente para  escolha de melhor estrutura.
As the simulation of the closed-loop (or even open-loop) response is not viable in DDC strategy, due to a lack of a model for the process, is intend to make use of Reinforcement Learning (RL) techniques, 
discussed in Chapter \ ref {cap: RL},
so that data collected from the real process, while in operation, can be used for real-time calculation of RIPs and consequently for choosing the best structure.

% Algumas técnicas de RL se mostram promissoras neste caso, uma vez que muitas vezes levam à otimização de índices de desempenho a partir de dados amostrados, sem a necessidade do modelo do processo, ao mesmo tempo em que evitam o alto número de realizações amostrais como em metodologias Monte Carlo. Dentre elas a estratégia TD Learning,
Some RL techniques are promising in this case, since they often lead to the optimization of performance indexes based on sampled data, without the need for the process model, while avoiding the high number of sample achievements as in Monte Carlo methodologies. Among them, the TD Learning strategy \citep{sutton2018},
\todo{Colocar referência}
% tem sido considerada, com uma atenção ao método conhecido como Q-learning \cite{watkins1992}, que permite que um controlador possa ser ajustado em uma abordagem conhecida como \textit{off-policy} que permite o aprendizado de uma política de controle enquanto o sistema em malha fechada se encontra sob o efeito de outra. No âmbito deste trabalho, o sentimento é que deva ser possível o uso de uma estratégia parecida juntamente com o concito do RAmSS, possa se atualizar os RIPs e consequentemente a estrutura, e talvez até parâmetros, do controlador com um menor esforço e, se possível, com provas de convergência.
has been considered, with attention to the method known as Q-learning \cite{watkins1992}, which allows a controller to be adjusted in an approach known as \textit{off-policy}, that allows the learning of a control policy while the closed-loop system is under the influence of another. In the scope of this work, the feeling is that it should be possible to use a similar strategy together with the RAmSS concept, to be able to update the RIPs and, consequently, the structure, and perhaps even parameters, of the controller with less effort and, if possible, with evidence of convergence.

% Por fim, análises a respeito de robustez, na presença de ruídos de medição ou de processos devem ser também levadas em conta, principalmente em uma etapa final. Inclusão de termos de média móvel ao procedimento RaMSS tem sido cogitado como alternativa a reduzir a polarização durante a etapa de identificação, em alternativa ao uso de estratégias baseadas em variáveis instrumentais, comuns no procedimento VRFT.
% Finally, analyzes regarding robustness, in the presence of measurement or process noise must also be taken into account, especially in a final step. Inclusion of moving average terms to the RaMSS procedure has been considered as an alternative to reducing polarization during the identification stage, as an alternative to the use of strategies based on instrumental variables, common in the VRFT procedure.



\vspace{2cm}

% No atual estágio desta pesquisa, tem-se trabalhado com a ideia de um índice que quantifique o erro médio de rastreamento em função do modelo do controlador sintonizado por uma estratégia VRFT, de modo que alguma informação da resposta em malha fechada com o controlador analisado possa ser usada para  o cálculo dos índices de probabilidades de inclusão do regressor, ou RIP (vide seção de Revisão Bibliográfica).
% Porém, como a simulação da resposta em malha fechada (ou mesmo malha aberta) é inviável devido a uma falta de modelo para o processo, o que se estuda é o uso de técnicas de Aprendizado por Reforço (RL) para que dados colhidos do processo real, enquanto em funcionamento, possam ser usados para o cálculo em tempo real dos RIP e consequentemente para  escolha de melhor estrutura.

% Algumas técnicas de RL se mostram promissoras neste caso, uma vez que muitas vezes levam à otimização de índices de desempenho a partir de dados amostrados, sem a necessidade do modelo do processo, ao mesmo tempo em que evitam o alto número de realizações amostrais como em metodologias Monte Carlo. Dentre elas a estratégia TD Learning, tem sido considerada, com uma atenção ao método conhecido como Q-learning, que permite que um controlador possa ser ajustado em uma abordagem conhecida como off-policy. Nesse caso, um controlador ou lei de controle (no caso do presente trabalho, mais precisamente sua estrutura) pode ser determinado enquanto o sistema em malha fechada obedece a uma lei (ou política) de controle diferente. Isto ajuda em eventuais problemas de instabilidade e exploração.

% Durante o semestre tem sido desenvolvido um algoritmo no intuito de implementar as ideiais consideradas anteriormente. Parte do algoritmo desenvolvido por Retes (2018), tem sido aproveitada, assim como algoritmos implementados e desenvolvidos pelo grupo do MACSIN1, grupo do qual o autor faz parte desde o início do doutorado. O algoritmo tem sido desenvolvido na linguagem R, portanto traduções de algumas ferramentas já disponíveis no MACSIN em outras linguagens tiveram e estão tendo que ser adaptadas. Os resultados apresentados na seção “Análise e Processamento de Dados” foram obtidos, em sua maioria, a partir deste algoritmo.
