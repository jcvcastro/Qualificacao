% -*- tex-master: "qualificacao.tex" -*-
%!tex root = qualificacao.tex

\chapter*{Resumo}
\addcontentsline{toc}{chapter}{Resumo}

\vspace{-2cm} 


Em procedimentos convencionais o projeto de controladores é feito a partir de um modelo matemático que representa o processo a se controlar. Outra abordagem, que difere em essência da convencional, com estudos crescentes nas últimas décadas, é a de projeto de controladores baseado em dados (DDC do inglês Data-Driven Control). No DDC, o projeto do controlador não faz uso direta ou indiretamente do modelo do processo e todo o projeto é feito a partir de dados amostrados diretamente do processo. Grande parte das técnicas DDC são métodos iterativos baseados principalmente no método do gradiente para minimizar algum índice de custo. Contudo algumas técnicas, em especial a VRFT, do inglês Virtual Reference Feedback Tuning, permitem, por um procedimento em batelada, realizar a minimização deste índice a partir de técnicas usuais no âmbito da identificação de sistemas. Em geral estes procedimentos são feitos fixando-se uma estrutura para o controlador, e a partir de métodos de identificação e otimização procura-se pelo melhor conjunto de parâmetros para o controlador que resulte em um comportamento próximo ao definido por um modelo de referência. Nesta pesquisa procura-se por uma abordagem alternativa, em que a melhor estrutura do controlador seja selecinada no processo de sintonia do controlador.  Neste sentido tem-se buscado um método para auxilio na seleção da melhor estrutura do controlador a partir de uma estratégia de controle VRFT para sistemas não lineares. Esta seleção é feita por uma abordagem aleatorizada de seleção de estruturas de modelos recente no âmbito de identificação de sistemas e que é, neste trabalho, adaptada para lidar com identificação de controladores. Como um texto de qualificação, alguns resultados prévios alcançados são estudados, e propostas para continuidade do trabalho são apresentadas. 

\textbf{Palavras-chave:} Controle baseado em dados; Controle livre de modelo; Aprendizado por reforço; Seleção de estruturas; Sistemas não lineares.
