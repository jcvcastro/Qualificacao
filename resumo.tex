% -*- tex-master: "qualificacao.tex" -*-
%!tex root = qualificacao.tex

\chapter*{Resumo}
\addcontentsline{toc}{chapter}{Resumo}

\vspace{-2cm} 


Em procedimentos convencionais o projeto de controladores é feito a partir de um modelo matemático que representa o processo a se controlar.a outra abordagem, que difere em essência da convencional, com estudos crescentes nas últimas décadas, é a de projeto de controladores baseado em dados (DDC do inglês Data-Driven Control). No DDC, o projeto do controlador não faz uso direta ou indiretamente do modelo do processo e todo o projeto é feito a partir de dados amostrados diretamente do processo. Grande parte das técnicas DDC são métodos iterativos baseados principalmente no método do gradiente para minimizar algum índice de custo. Contudo algumas técnicas, em especial a VRFT, do inglês Virtual Reference Feedback Tuning, permitem, por um procedimento em batelada, realizar a minimização deste índice a partir de técnicas usuais no âmbito da identificação de sistemas. No contexto de identificação de sistemas é consenso que a estrutura do modelo – quais regressores o compõem – tem forte influência no desempenho dinâmico. Neste sentido esta pesquisa de doutorado tem buscado um método para auxilio na seleção da melhor estrutura do controlador a partir de uma estratégia de controle VRFT para sistemas não lineares. Esta seleção é feita por uma abordagem aleatorizada de seleção de estruturas de modelos já conhecida no âmbito de identificação de sistemas, mas que é, neste trabalho, adaptada para lidar com identificação de controladores. Afim de reduzir custo computacional e tornar o procedimento mais viável, o uso de técnicas de aprendizado por reforço é considerado. Por fim, o uso de informações auxiliares, que tem mostrado benefícios no contexto de identificação de sistemas, é analisado no contexto da identificação de controladores, por meio de restrições que levam em conta características previamente conhecidas no projeto do controlador.



\textbf{Palavras-chave:} Controle baseado em dados; Controle livre de modelo; Aprendizado por reforço; Seleção de estruturas; Sistemas não lineares.
