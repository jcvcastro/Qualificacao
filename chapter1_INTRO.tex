% -*- TeX-master: "Qualificacao.tex" -*-
%!TEX root = Qualificacao.tex

\chapter{Introduction}
\label{cap1} \vspace{-1cm}

% \begin{flushright}
% \begin{minipage}{0.7\linewidth}
% \emph{``...''}
% \end{minipage}
% \end{flushright}
%
% \begin{flushright}
% {fulano}
% \end{flushright}
The use of feedback control in mechanisms developed by humans is marked by the 1769 James Watt's invention, known as the Watt regulator and developed to regulate steam-machines spin velocities.
% The emergence of systems control using feedback is marked by 1769 with the invention of James Watt, known as the Watt regulator.
From this time until the beginning of the 20th century, control designs were based on trial and error methods. With the emergence of theoretical publications on the subject, such as that of \citep{tolle1921}, mathematical models were increasingly used in the design of controllers, mainly in the form of differential equations \citep{takahashi1972}.

In the 30s and 50s the so-called Classical Control Theory originates, expressing itself basically in the frequency domain and in the $s$-plane, with models given by transfer functions, based on methods developed mainly by Nyquist, Bode, Nichols and Evans.

In the 1960s, a new control theory approach arises, using parametric models and state space representation. this approaches gives rise to the so-called Modern Control Theory and its main branches, such as systems identification, adaptive control, robust control, optimal control and stochastic control, which have been widely studied and developed until today, but still with many challenging topics, both in theoretical and practical aspects \citep{hou2013}.

In both approaches, the classical control theory, mainly based on the use of transfer functions and linear systems, as in the modern control theory, mainly based on state space representations of linear and non-linear systems, a mathematical model of the process to be controlled is required
\todo{colocar referencia para trabalhos relevantes do tipo, para a footnote.}\footnote{with some exceptions like the cases where the controller is designed directly from the frequency response obtained experimentally.}.
Such model can be obtained via phenomenological modeling, or via systems identification methods. In the former case, the model is obtained using known laws from specific fields of science resulting in equations that represent it. In the latter case, using input-output data collected from the process and using systems identification techniques, models that represent the process are obtained, with a certain degree of reliability.

Several methodologies for identifying linear and non-linear models are available in the literature \citep{aguirre2015, ljung1999}.

Models obtained using first principles or even by systems identification can result in high order models, with a high degree of non-linearity, which makes difficult or even impractical their use for control purposes.

% For cases where the order of the model is high, an order reduction step must often be included in the \citep{skogestad1997} control project.


Furthermore, modelling processes can be an arduous task and sometimes even impracticable, requiring steps to validate and determine the structure of the model. \todo{colocar referencias de trabalhos aqui?}

For this reason, traditional model-based control methods (MBC) are unpractical in some cases. Besides, several processes generate and store large amounts of data and the use of this data for controller design would be very convenient \citep{hou2013}.

Since the input and output data of a plant contains information about its dynamics, as long as it is properly excited, it may seem unnecessary to apply the identification theory to obtain a mathematical model of the plant for controller design \citep{ikeda2001}.
Furthermore, in an attempt to obtain a model that is faithful to the behaviour of the process, a very complex model can be arrived at, and a process of order reduction may be necessary during the controller design. In this case, additional effort in identifying the model may be unnecessary when designing the controller.
% Also, having obtained a model faithful to the plant, it may be necessary to reduce its order in the design of the controller.

In this sense, in several practical control cases in which a mathematical model describing the plant is not available, or is too complex or the uncertainty in the model is too great for the use of MBC strategies, it is very convenient to obtain the controller from measurements obtained directly from the plant.

According to \cite{campi2002}, this problem has attracted the attention of control engineers since the work published by \cite{ziegler1942}, and several extensions have been proposed since then.
Such procedures, despite being similar to trial and error procedures, were widely used in the industry, perhaps due to their simplicity of design, even if at the expense of final performance losses.

Around the 1990s, new approaches to controller design without the use of models for plants began to appear in the literature, which later came to be called control based on data (DDC).
%
\cite{hou2013} claim that the term \emph{data-driven} was first proposed in computer science and only recently entered the vocabulary of the control community and, to date, there are some DDC methods knouwn by different names, such as `` \emph{data-driven control} '', `` \emph{data-based control} '', `` \emph{modeless control} '', among others. \cite{hou2013} propose the following definition for DDC, based on 3 other definitions:


\begin{defn}[Data-Driven Control]\citep{hou2013}
Data-driven control includes all control theories and methods in which the controller is designed by directly using on-line or off-line I/O data of the controlled system or knowledge from the data processing but not any explicit information from mathematical model of the controlled process, and whose stability, convergence, and robustness can be guaranteed by rigorous mathematical analysis under certain reasonable assumptions.
\end{defn}

Therefore, the DDC is different from the MBC in essence, since the controller design does not make direct or indirect use of the process model.  
Although at first, they look like adaptive control methods, DDC methods differ from these in that, at first, they do not need any model information, and parameter settings depend on large batches of data, instead of only a few samples of the input-output signals. % \citep{bazanella2012}.


	\todo{Uma perspectiva do desenvolvimento do assunto na comunidade acadêmica pode ser obtida por uma busca
		% \footnote{os resultados obtidos foram refinados pelas seguintes categorias do Web of Science: ``\emph{automation control systems}'' e ``\emph{engineering electrical electronic}'', que se mostraram as mais significativas.}
		pelo número de publicações na base de dados do \cite{webofscience2018}, utilizando o termo ``\emph{data-driven control}'' e pela combinação de termos ``\emph{data-driven} or \emph{data-based control} or \emph{modeless control} or \emph{model-less control}'', e seu resultado é apresentado na Figura~\ref{Fig_1}.
		Percebe-se um crescente aumento no número de publicações e citações ao longo dos anos.
	}

Some conceptually distinct approaches using DDC appear in the literature in the last years, among them\footnote{it was chosen here to mention some techniques that the author found most relevant to this proposal, however others can be found in the literature \citep{spall1992, safonov1995, karimi2007, huang2008, schaal1994, shi2000}}:
\emph{Virtual Reference Feedback Tuning} (VRFT), Iterative Feedback Tuning (IFT), \emph{Frequency Domain Tuning} (FDT), \emph{Correlation Based Tuning} (CbT), originally presented by \cite{campi2002}, \cite{hjalmarsson1994}, \cite{kammer2000} and \cite{karimi2002}, respectively.

Since the input and output data of a plant contains information about its dynamics, as long as it is properly excited, it may seem unnecessary to apply the identification theory to obtain a mathematical model of the plant for controller design \citep{ikeda2001}.
In addition, having obtained a model faithful to the plant, it may be necessary to reduce its order in the design of the controller.
In this sense, in several practical control cases in which a mathematical model describing the plant is not available, or is too complex or the uncertainty in the model is too great for the use of MBC strategies, it is very convenient to obtain the controller from measurements obtained directly from the plant.

According to \cite{campi2002}, this problem has attracted the attention of control engineers since the work published by \cite{ziegler1942} and several extensions have been proposed since then. However, around the 1990s, new approaches to controller design without the use of models for plants began to appear in the literature, which later came to be called control based on data \emph{(DDC - from English, data-driven control)}.
%
\cite{hou2013} claim that the term \emph{data-driven} was first proposed in computer science and only recently entered the vocabulary of the control community and, to date, there are some DDC methods, however they are characterized by different names, such as `` \emph{data-driven control} '', `` \emph{data-based control} '', `` \emph{modeless control} '', among others. \cite{hou2013} propose the following definition for DDC, based on 3 other definitions found on the Internet:

\begin{defn}[Data-Driven Control]\citep{hou2013}
Data-driven control includes all control theories and methods in which the controller is designed by directly using on-line or off-line I/O data of the controlled system or knowledge from the data processing but not any explicit information from mathematical model of the controlled process, and whose stability, convergence, and robustness can be guaranteed by rigorous mathematical analysis under certain reasonable assumptions.
\end{defn}


Therefore, the DDC is different from the MBC in essence, since the controller design does not make direct or indirect use of the process model.  
Although at first, they look like adaptive control methods, DDC methods differ from these in that, at first, they do not need any model information, and parameter settings depend, in general, on large batches of data, instead of only a few samples of the input-output signals. % \citep{bazanella2012}.

	\todo{Uma perspectiva do desenvolvimento do assunto na comunidade acadêmica pode ser obtida por uma busca
		% \footnote{os resultados obtidos foram refinados pelas seguintes categorias do Web of Science: ``\emph{automation control systems}'' e ``\emph{engineering electrical electronic}'', que se mostraram as mais significativas.}
		pelo número de publicações na base de dados do \cite{webofscience}, utilizando o termo ``\emph{data-driven control}'' e pela combinação de termos ``\emph{data-driven} or \emph{data-based control} or \emph{modeless control} or \emph{model-less control}'', e seu resultado é apresentado na Figura~\ref{Fig_1}.
		Percebe-se um crescente aumento no número de publicações e citações ao longo dos anos.
	}

Some conceptually distinct approaches using DDC appear in the literature in the last years, among them\footnote{it was chosen here to mention some techniques that the author found most relevant to this proposal, however others can be found in the literature \citep{sadegh1998, safonov1995, karimi2007, huang2008, schaal1994, shi2000}}:
\emph{Virtual Reference Feedback Tuning} (VRFT), Iterative Feedback Tuning (IFT), \emph{Frequency Domain Tuning} (FDT), \emph{Correlation Based Tuning} (CbT), originally presented by \cite{campi2002}, \cite{hjalmarsson1994}, \cite{kammer2000} and \cite{karimi2002}, respectively.
Most of these methodologies use the concept of optimization from the minimization of a cost function, in general, measured in terms of the $H_2$ norm of a signal. 
Several DDC methods available in the literature do this optimization in an iterative way, among them, the IFT, CbT, ILC, ADP. 
Others do so in batches, such as the VRFT and \textit{Noniterative data-driven model reference control} methods. \todo{Falar um pouco, ou pelo menos citar o \textit{Noniterative data-driven model reference control}} 

In iterative cases, the minimization of the cost function is done, typically, by gradient descent methodologies, from input-output data collected in a batch way \citep{bazanella2008a}.
One drawback of these methodologies is the lack of conditions that guarantee convergence to a global minimum for the cost function in many cases.
% In this sense, \cite{huusom2009} present a study that extends the IFT method in order to improve the convergence properties and reduce the number of process experiments required by IFT.
Extensions to improve the convergence properties and even reduce the number of required in-process experiments have been the subject of studies in last years \citep{huusom2009}.

In non-iterative cases, convergence to a global minimum is generally not an issue. 
The VRTF method, presented by \cite{guardabassi2000a, campi2002} to deals with the design of SISO systems and results in a linear controller is an example.
In order to solve the problem of convergence to a global minimum of a $ H_2 $ performance criterion, the VRFT focus on making the cost function be optimized sufficiently ``well behaved'' making optimization converge properly.

At first, given ideal conditions, convergence to the global minimum is not a problem when using the VRFT method, as it is a batch method. 
In addition, VRFT has no initialization problems and does not access the plant several times for experimentation, in contrast to iterative methods, allowing to maintaining the normal process operation. 
Extensions for non-linear controllers designs have been proposed since then \citep{campi2006a, jeng2014a, jeng2018a}.


\todo[inline]{Vou terminar aqui ainda. Levar para o lado da seleção de estrutura. Se for preciso, resumo um pouco o texto anterior.} 
% >>>>  CHECKED WITH GRAMMARLY UNTIL HERE <<<<--------------------------------------------------

\newpage
% The VRTF method in its first versions presented by \cite{guardabassi2000, campi2002}, deals with the design of SISO systems and results in a linear controller.
	%\todo{olhar a questão da anotação do Aguirre: ``reservar a frase''}Assim como o caso anterior, a tarefa de modelar por identificação em geral não é fácil, exigindo etapas de validação e determinação de estrutura do modelo. 
	%
	% Modelos obtidos utilizando primeiros princípios ou mesmo por identificação de sistemas podem resultar em modelos de ordem  elevada, de alto grau de não linearidade o que dificulta ou até mesmo impede sua aplicação para fins de controle.
	%
	% % Um exemplo é trabalho de \citep{Martins2016} que usa uma abordagem para obtenção de um modelo final que apresenta vantagens do pondo de vista de controle sobre outros modelos obtidos pela abordagem fenomenológica.
	% Para casos em que a ordem do modelo é elevada, muitas vezes uma etapa de redução de ordem deve ser incluída no projeto de controle \citep{skogestad1997}.
	%
	% Modelar processos pode ser uma tarefa árdua e às vezes até impraticável podendo exigir etapas de validação e determinação de estrutura do modelo.
	% %
	% Por esta razão os métodos tradicionais de controle baseados em modelo (MBC, do inglês \emph{model based control}) são pouco práticos em alguns casos. Além do mais, vários processos da atualidade geram e armazenam grandes quantidades dados e o uso desses dados para projeto de controladores seria muito conveniente \citep{hou2013}.
	%
	% Uma vez que os dad os de entrada e saída de uma planta contêm informações sobre sua dinâmica, desde que excitada apropriadamente, pode parecer desnecessário aplicar a teoria de identificação para se obter um modelo matemático da planta para projeto do controlador \citep{ikeda2001}.
	% Além disso, tendo obtido um modelo fiel à planta pode ser necessário reduzir sua ordem no projeto do controlador.
	% %
	% Neste sentido, em vários casos práticos de controle em que um modelo matemático que descreve a planta não está disponível, ou é complexo demais ou a incerteza no modelo é grande demais para o uso de estratégias MBC, é muito conveniente obter o controlador a partir de medidas obtidas diretamente da planta.
%
	% De acordo com \cite{campi2002}, este problema tem atraído atenção de engenheiros de controle desde o trabalho publicado por \cite{ziegler1942} e diversas extensões têm sido propostas desde então. Porém, por volta da década de 90 começam a surgir na literatura novas abordagens para projeto de controladores sem o uso de modelos para as plantas, as quais mais tarde vêm a receber denominação de controle baseado em dados \emph{(DDC - do inglês, data-driven control)}.
	% %
	% \cite{hou2013} afirmam que o termo \emph{data-driven} foi primeiramente proposto na ciência da computação e somente recentemente entrou no vocabulário da comunidade de controle sendo que, até o momento, existem alguns métodos DDC, porém são caracterizados por diferentes nomes, como ``\emph{data-driven control}'', ``\emph{data-based control}'', ``\emph{modeless control}'', dentre outros. \cite{hou2013} propõem a seguinte definição para DDC, a partir de 3 outras definições encontradas na Internet:
	
	% \begin{citacao}
		% ``O controle baseado em dados inclui todas as teorias e métodos de controle nos quais o controlador é projetado usando diretamente dados de E/S \emph{on-line} ou \emph{off-line} do sistema controlado ou conhecimento do processamento de dados, mas nenhuma informação explícita do modelo matemático do processo controlado, e cuja estabilidade, convergência e robustez podem ser garantidas por rigorosa análise matemática sob certas suposições razoáveis.''\citep[p.~6, tradução livre]{hou2013}.
		%\footnote{``Data-driven control includes all control theories and methods in which the controller is designed by directly using on-line or off-line I/O data of the controlled system or knowledge from the data processing but not any explicit information from mathematical model of the controlled process, and whose stability, convergence, and robustness can be guaranteed by rigorous mathematical analysis under certain reasonable assumptions.''}\todo{Dúvida: colocar ou não o termo original em inglês na nota de rodapé?}
	% \end{citacao}

	% Portanto, o DDC é diferente do MBC em essência, uma vez que o projeto do controlador não faz uso direta o indiretamente do modelo do processo
	% \todo[color=cyan]{apesar do uso de uma aproximação dos modelos de processo como modelos secundários em algumas metodologias}
	 
	% Apesar de, em um primeiro momento, parecerem com métodos de controle adaptativos, métodos DDC diferem destes pelo fato de, a princípio, não precisam de nenhuma informação do modelo, e os ajustes dos parâmetros dependem de grandes lotes de dados, ao invés de uma única o poucas amostras do sinais de entrada-saída \citep{bazanella2012}.

	
	% Uma perspectiva do desenvolvimento do assunto na comunidade acadêmica pode ser obtida por uma busca
	% % \footnote{os resultados obtidos foram refinados pelas seguintes categorias do Web of Science: ``\emph{automation control systems}'' e ``\emph{engineering electrical electronic}'', que se mostraram as mais significativas.}
	% pelo número de publicações na base de dados do \cite{webofscience}, utilizando o termo ``\emph{data-driven control}'' e pela combinação de termos ``\emph{data-driven} or \emph{data-based control} or \emph{modeless control} or \emph{model-less control}'', e seu resultado é apresentado na Figura~\ref{Fig_1}.
	% Percebe-se um crescente aumento no número de publicações e citações ao longo dos anos.

	%, a começar por 1996, com os trabalhos de \cite{Chan1996,chan1996experimental}, que propõe um método numérico para a síntese de um controlador em malha fechada a partir de dados obtidos de uma planta de fase mínima em malha aberta sem o conhecimento explícito de seu modelo paramétrico.

	% \begin{figure}[!htbb]
		% \centering
		% \includegraphics [scale=1]{grafico_1.pdf}
		% % \missingfigure{Colocar aqui gráfico do WebofScience}
		% \setlength{\belowcaptionskip}{-12pt}
		% \caption[Número de publicações]{Números de publicações anuais ao se usar a palavras-chave ``\emph{data-driven control}'' (escuro) e a combinação das palavras chaves  ``\emph{data-driven} or \emph{data-based control} or \emph{modeless control} or \emph{model-less control}'' (claro) na base de dados \cite{webofScience}.}
		% \label{Fig_1}
	% \end{figure}

	% Algumas abordagens conceitualmente distintas utilizando DDC aparecem na literatura, dentre elas\footnote{os nomes dos métodos, em geral batizados por seus autores, foram mantidos na linguagem original por muitas vezes não terem uma tradução ainda difundida na literatura brasileira.}, citam-se\footnote{optou-se aqui por citar algumas técnicas que o autor achou mais relevantes para esta proposta, porém outras podem ser encontradas na literatura \citep{sadegh1998,safonov1995,karimi2007,huang2008,schaal1994,shi2000}}:
	% \emph{Virtual Reference Feedback Tuning} (VRFT), Iterative Feedback Tuning (IFT), \emph{Frequency Domain Tuning} (FDT), \emph{Correlation Based Tuning} (CbT), apresentadas originalmente por \cite{campi2002}, \cite{hjalmarsson1994}, \cite{kammer2000} e \cite{karimi2002}, respectivamente.

	% \todo{Dúvida: usar funcional de custo ou critério de desempenho?\\Resposta do Aguirre: Sáo coisas um pouco diferentes. No contexto de otimização use ``funcional'', no contexto de validação use ``creitério de desempenho''.}
	% A maioria destas metodologias utilizam o conceito de otimização a partir da minimização de um funcional de custo, em geral medido em termos da norma $H_2$ de um sinal da malha. Vários métodos DDC disponíveis na literatura fazem esta otimização de maneira iterativa, dentre eles, o IFT, CbT, ILC, ADP. Outros, o fazem em batelada, como o caso dos métodos VRFT e \emph{Noniterative data-driven model reference control}.

	% \todo[color=cyan]{Gradiente descendente ou método do gradiente? Aguirre sugeriu ``método do gradiente''  mas está difícil encaixar.}gradiente descendente
	% Nos casos iterativos, a minimização da funcional de custo é feita, tipicamente, pelo método do gradiente, a partir de dados de entrada-saída coletados em batelada \citep{bazanella2008}.
	% Um problema na aplicação destes métodos é a falta de condições que garantem convergência para um mínimo global para o índice de desempenho ao se usar estes algoritmos.
	% Afim de resolver o problema da convergência para um mínimo global de um critério de desempenho $H_2$, \cite{bazanella2008} focam em fazer com que a função de custo a ser otimizada seja suficientemente ``bem comportada'' fazendo com que qualquer algoritmo (correto) de otimização convirja propriamente.
	% Outro trabalho neste sentido é o de \cite{huusom2009} que estendem o método IFT afim melhorar as propriedades de convergência e reduzir o número de experimentos requeridos com a planta.

	% A princípio, considerando condições ideais, a convergência para o mínimo global não é problema quando se utiliza o método VRFT, por se tratar de um método a batelada.
	% Além do mais, este método não apresenta problemas de inicialização e não acessa a planta várias vezes para experimento, ao contrário de métodos iterativos, mantendo sua operação normal.
	% O método VRTF em suas primeiras versões apresentadas por \cite{guardabassi2000,campi2002}, lida com o projeto de sistemas SISO e resulta em um controlador linear.
	% Extensões para o caso de controladores não lineares têm sido propostas desde então \citep{campi2007,jeng2014,jeng2018}.


% ====================================================================================================
	% Devido a suas características atrativas pretende-se, neste trabalho, pelo menos a princípio, utilizar a abordagem VRFT.
	% Esta abordagem formula o problema de sintonia do controlador como um problema de identificação via a introdução de um sinal virtual de referência \citep{hou2013}.
	% O objetivo de controle é minimizar um funcional de custo dado pela norma $H_2$ da diferença entre função de transferência em malha fechada e um modelo de referência, ambos multiplicados pelo sinal de referência $r$.
	% O problema em achar o mínimo é que não há modelo disponível, impedindo o cálculo do modelo em malha fechada.
	% Visando contornar este problema, o conceito de sinais virtuais é usado.
	% Estes sinais, dados por $e^{vir}$ (erro virtual) e $u^{vir}$ (sinal de controle virtual), são criados a partir do sinal de saída da planta  e do modelo de referência inverso, possibilitando o uso de um novo funcional de custo dado por $J_{vir}=||C(\theta,z^{-1})e^{vir}-u^{vir}||$, em que $C(\theta,z^{-1})$ representa o modelo do controlador cujos parâmetros $\theta$ devem ser identificados por otimização.
	% %
	% \cite{campi2002} mostram que ao minimizar $J_{vir}$, minimiza-se o primeiro critério sob certas condições. A minimização do novo funcional pode ser feita por técnicas de estimadores de mínimos quadrados (MQ), variáveis instrumentais (VI), dentre outras \citep{aguirre2015}. \cite{bazanella2012} mostram exemplos do uso de variáveis instrumentais para resolver o problema de polarização dos parâmetros identificados para casos de sinais ruidosos.
%
	% %
	% Até o momento, com base na literatura, encontrou-se técnicas que estendem a abordagem VRFT para casos não lineares \citep{previdi2004a,campi2006a,wang2011a,bazanella2014a,yan2016a,radac2018b}. Mas de acordo com \cite{jeng2018a}, diferentemente do VRFT linear, estas versões estendidas para sistemas não lineares ou não são em batelada, ou suas soluções não podem ser determinadas por métodos MQ, perdendo uma vantagem considerável do VRFT. Porém \cite{jeng2015a} mostram que o VRFT pode ser estendido para o controle de sistemas não lineares do tipo Hammerstein e Wiener de forma que é mantido a característica não iterativa do VRFT. Três anos depois, \cite{jeng2018a} apresentam um método onde somente a não linearidade estática (ou sua inversa), representada por séries $B$\emph{-spline}, é estimada simultaneamente com o controlador sem a necessidade de otimização não linear ou procedimentos iterativos.
%
	% Uma pergunta que surge é: seria possível o uso de técnicas de estimação do tipo MQ com restrições para modelos não lineares NARX (do inglês \emph{Nonlinear model with eXogenous inputs}) ou MQ estendido para modelos NARMAX (do inglês \emph{Nonlinear AutoRegressive Moving Average model with eXogenous inputs}) neste tipo de abordagem?
	% % porém pouca coisa se encontrou a respeito do uso de estimadores não lineares com restrições para casos em que tem-se um prévio conhecimento do modelo (por exemplo para sistemas não lineares que apresentam comportamento histerético como o tratado em \cite{Martins2016}),  configurando um problema do tipo ``caixa cinza''. \todo[color=red]{falar a respeito dos trabalhos...}\cite{Jeng2014,Jeng2015,Jeng2018} apresentam trabalhos neste sentido, mas foram os únicos encontrados pelo autor até o momento. Portanto o uso de técnicas identificação não lineares baseadas em MQ com restrições é um ponto importante a ser investigado.
	% % Até o momento, com base na literatura, encontrou-se técnicas que estendem a abordagem VRFT para casos não lineares \cite{Previdi2004,campi2006,Wang2011,Yan2016,Radac2018}, porém pouca coisa se encontrou a respeito do uso de estimadores não lineares com restrições para casos em que tem-se um prévio conhecimento do modelo (por exemplo para sistemas não lineares que apresentam comportamento histerético como o tratado em \cite{Martins2016}),  configurando um problema do tipo ``caixa cinza''. \todo[color=red]{falar a respeito dos trabalhos...}\cite{Jeng2014,ISI:000366889700127,Jeng2018} apresentam trabalhos neste sentido, mas foram os únicos encontrados pelo autor até o momento. Portanto o uso de técnicas identificação não lineares baseadas em MQ com restrições é um ponto importante a ser investigado.
	% Apesar de já terem sido desenvolvidas técnicas para incorporar informação auxiliar no processo de identificação, por exemplo via restrições e otimização multiobjetivo \citep{barroso2006}, todas estas restrições dizem respeito à planta.
	% Neste sentido surgem questões como: de que forma estas técnicas podem ser usadas na abordagem DDC?
	% %
	% % Seria possível encontrar um análogo da informação auxiliar, obtida para métodos tradicionais a partir de informações da planta, para uso em estratégias DDC, que não têm modelo da planta disponível, por exemplo, a partir restrições para garantir aspectos relevantes ao controle como limitações de ganho devido a saturação de atuadores, ou ate mesmo relativos a robustez?
	% %
	% Seria possível encontrar um análogo da informação auxiliar, usada em métodos tradicionais, para estratégias DDC, em que não há informação da planta?
	% Poderia esta ser definida, por exemplo, a partir restrições que garantam aspectos relevantes ao controle, como limitações de ganho devido a saturação de atuadores, ou até mesmo relativos a robustez?
