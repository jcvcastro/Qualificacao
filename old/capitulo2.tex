%------------------------------------------------------------------------------
\chapter{Reinforcement Learning}
\vspace{-1cm} \label{cap2}

% \begin{flushright}
% \begin{minipage}{0.7\linewidth}
%     \emph{``\dots''}
% \end{minipage}
% \end{flushright}
%
% \begin{flushright}
% Cicrano
% \end{flushright}

%\vspace{1cm}

%Colocar uma descri��o do cap�tulo aqui!
\section{Introduction}\label{sec:int_cap_2}
\section{Approximation in Value Space}\label{}
\begin{itemize}
    \item Suboptimal control;
    \item $J_k^{*}$ is approximated by $\tilde{J}_k$;
    \item At state $k$ we use control obtained from
        \begin{equation}
        \tilde{\mu}_k(x_k) \in \text{arg} \min_{u_k \in U_k(x_k)}{E\left\{ g_{k}(x_k,\ u_k,\ g_k) + J_{k+1}\left(f_k(x_k,\ u_k,\ w_k)\right) \right\}} 
        \label{e:201}.
        \end{equation}
\end{itemize}
The approximation in value space~\eqref{e:201} is referred as a one step lookahead equation.
A variation can be accomplished by taking a multi-step lookahead approach, where the cost function 
% TODO: olhar se � a cost function que � minimizada mesmo.
is minimized over $l+1$ stages with future costs approximated by a function $\tilde{J}_{k+l}$ given by
\begin{equation}
\tilde{J}_{k+1}(x_{k+1}) = \min_{u_{k+1} \in U_{k+1}(x_{k+1})} E\left\{ g_{k+1}(x_{k+1},\ u_{k+1},\ w_{k+1}) + \tilde{J}_{k+2}\left( f_{k+1}(x_{k+1},\ u_{k+1},\ w_{k+1}) \right)  \right\}
\label{e:Jkp1}.
\end{equation}

\section{Rollout}
% Section 2.4 - Bertsekas
\begin{itemize}
    \item Principal aim: policy improvement
    \item ``Approximation in value space with the approximate costo-to-go values $\tilde{J}_{k+l}(x_{k+l})$ calculated by running the base policy.''
    \item ``choice of base policy is important, but experimental evidences has shown that it may not be crucial for many contexts''
\end{itemize}

% Sectionn 2.4.1 - On-line Rollout for Deterministic Finite-State Problems
\begin{itemize}
    \item ``Considers every tail subproblem starting at every possible next state $x_{k+1}$ solving them using a suboptimally algorithm, the \emph{base heuristics}.''
    \item Rollout algorithm then applies a control that minimizes over $u_k \in U_k(x_k)$ the tail expression
        \begin{equation}
            \tilde{v} g_k(x_k,u_k) +  g_{k+1}(x_{k+1}) + \cdots + g_{N-1}(x_{N-1},u_{N-1}) + g_N(x_N)
            \label{eq:rollout}.
        \end{equation}
\end{itemize}
